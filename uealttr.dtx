% \iffalse meta-comment
%
% Copyright (C) 2008 by
%    Joseph Wright <joseph.wright@uea.ac.uk>
%
% This work may be distributed and/or modified under the
% conditions of the LaTeX Project Public License, either
% version 1.3c of this license or (at your option) any later
% version. The latest version of this license is in
%    http://www.latex-project.org/lppl.txt
% and version 1.3c or later is part of all distributions of
% LaTeX version 2005/12/01 or later.
%
% This work has the LPPL maintenance status `maintained'.
%
% The current maintainer of this work is Joseph Wright.
%
% This work consists of the source file uealttr.dtx
%                 and the derived files uealttr.ins,
%                                       uealttr.cls,
%                                       jawltxdoc.sty,
%                                       uealttr-manual.pdf and
%                                       uealttr.pdf
%
% TDS-ready files:
%    The compressed file uealttr.tds.zip contains an unpacked version
%    of all of the files included here, and pre-compiled
%    documentation in PDF format.  Simply decompress uealttr.tds.dtx
%    in your local TeX directory, run your hash program (texhash,
%    initexmf --update-fndb, etc.) and everything will be ready to
%    go.  The user documentation for the package is called
%    uealttr-manual.pdf; the file uealttr.pdf includes the user
%    manual and the full source code.
%
% Unpacking:
%    (a) If uealttr.ins is present:
%           tex uealttr.ins
%    (b) Without uealttr.ins:
%           tex uealttr.dtx
%    (c) If you use LaTeX to generate files:
%           latex \let\install=y% \iffalse meta-comment
%
% Copyright (C) 2008 by
%    Joseph Wright <joseph.wright@uea.ac.uk>
%
% This work may be distributed and/or modified under the
% conditions of the LaTeX Project Public License, either
% version 1.3c of this license or (at your option) any later
% version. The latest version of this license is in
%    http://www.latex-project.org/lppl.txt
% and version 1.3c or later is part of all distributions of
% LaTeX version 2005/12/01 or later.
%
% This work has the LPPL maintenance status `maintained'.
%
% The current maintainer of this work is Joseph Wright.
%
% This work consists of the source file uealttr.dtx
%                 and the derived files uealttr.ins,
%                                       uealttr.cls,
%                                       jawltxdoc.sty,
%                                       uealttr-manual.pdf and
%                                       uealttr.pdf
%
% TDS-ready files:
%    The compressed file uealttr.tds.zip contains an unpacked version
%    of all of the files included here, and pre-compiled
%    documentation in PDF format.  Simply decompress uealttr.tds.dtx
%    in your local TeX directory, run your hash program (texhash,
%    initexmf --update-fndb, etc.) and everything will be ready to
%    go.  The user documentation for the package is called
%    uealttr-manual.pdf; the file uealttr.pdf includes the user
%    manual and the full source code.
%
% Unpacking:
%    (a) If uealttr.ins is present:
%           tex uealttr.ins
%    (b) Without uealttr.ins:
%           tex uealttr.dtx
%    (c) If you use LaTeX to generate files:
%           latex \let\install=y% \iffalse meta-comment
%
% Copyright (C) 2008 by
%    Joseph Wright <joseph.wright@uea.ac.uk>
%
% This work may be distributed and/or modified under the
% conditions of the LaTeX Project Public License, either
% version 1.3c of this license or (at your option) any later
% version. The latest version of this license is in
%    http://www.latex-project.org/lppl.txt
% and version 1.3c or later is part of all distributions of
% LaTeX version 2005/12/01 or later.
%
% This work has the LPPL maintenance status `maintained'.
%
% The current maintainer of this work is Joseph Wright.
%
% This work consists of the source file uealttr.dtx
%                 and the derived files uealttr.ins,
%                                       uealttr.cls,
%                                       jawltxdoc.sty,
%                                       uealttr-manual.pdf and
%                                       uealttr.pdf
%
% TDS-ready files:
%    The compressed file uealttr.tds.zip contains an unpacked version
%    of all of the files included here, and pre-compiled
%    documentation in PDF format.  Simply decompress uealttr.tds.dtx
%    in your local TeX directory, run your hash program (texhash,
%    initexmf --update-fndb, etc.) and everything will be ready to
%    go.  The user documentation for the package is called
%    uealttr-manual.pdf; the file uealttr.pdf includes the user
%    manual and the full source code.
%
% Unpacking:
%    (a) If uealttr.ins is present:
%           tex uealttr.ins
%    (b) Without uealttr.ins:
%           tex uealttr.dtx
%    (c) If you use LaTeX to generate files:
%           latex \let\install=y% \iffalse meta-comment
%
% Copyright (C) 2008 by
%    Joseph Wright <joseph.wright@uea.ac.uk>
%
% This work may be distributed and/or modified under the
% conditions of the LaTeX Project Public License, either
% version 1.3c of this license or (at your option) any later
% version. The latest version of this license is in
%    http://www.latex-project.org/lppl.txt
% and version 1.3c or later is part of all distributions of
% LaTeX version 2005/12/01 or later.
%
% This work has the LPPL maintenance status `maintained'.
%
% The current maintainer of this work is Joseph Wright.
%
% This work consists of the source file uealttr.dtx
%                 and the derived files uealttr.ins,
%                                       uealttr.cls,
%                                       jawltxdoc.sty,
%                                       uealttr-manual.pdf and
%                                       uealttr.pdf
%
% TDS-ready files:
%    The compressed file uealttr.tds.zip contains an unpacked version
%    of all of the files included here, and pre-compiled
%    documentation in PDF format.  Simply decompress uealttr.tds.dtx
%    in your local TeX directory, run your hash program (texhash,
%    initexmf --update-fndb, etc.) and everything will be ready to
%    go.  The user documentation for the package is called
%    uealttr-manual.pdf; the file uealttr.pdf includes the user
%    manual and the full source code.
%
% Unpacking:
%    (a) If uealttr.ins is present:
%           tex uealttr.ins
%    (b) Without uealttr.ins:
%           tex uealttr.dtx
%    (c) If you use LaTeX to generate files:
%           latex \let\install=y\input{uealttr.dtx}
%
% Documentation:
%    (a) Without write18 enabled:
%          pdflatex uealttr.dtx
%          makeindex -s gind.ist uealttr.idx
%          makeindex -s gglo.ist -o uealttr.gls uealttr.glo
%          pdflatex uealttr.dtx
%          pdflatex uealttr.dtx
%    (b) With write18 enabled:
%          pdflatex uealttr.dtx
%          pdflatex uealttr.dtx
%          pdflatex uealttr.dtx
%
% Installation:
%     Copy uealttr.sty to a location searched by TeX, and if required
%     by your TeX installation, run the appropriate command to build
%     a hash of files (texhash, initexmf --update-fndb, etc.)
%
% Note:
%     The jawltxdoc.sty file is not needed for installation, only for
%     building the documentation; it may be deleted after producing
%     the documentation (if necessary).
%
%<*ignore>
% This is all taken verbatim from Heiko Oberdiek's packages
\begingroup
  \def\x{LaTeX2e}%
\expandafter\endgroup
\ifcase 0\ifx\install y1\fi\expandafter
         \ifx\csname processbatchFile\endcsname\relax\else1\fi
         \ifx\fmtname\x\else 1\fi\relax
\else\csname fi\endcsname
%</ignore>
%<*install>
\input docstrip.tex
\keepsilent
\askforoverwritefalse
\preamble
 ----------------------------------------------------------------
 The uealttr package --- A letter class for UEA
 Maintained by Joseph Wright
 E-mail: joseph.wright@uea.ac.uk
 Released under the LaTeX Project Public License v1.3c or later
 See http://www.latex-project.org/lppl.txt
 ----------------------------------------------------------------

\endpreamble
\Msg{Generating uealttr files:}
\generate{\file{jawltxdoc.sty}{\from{\jobname.dtx}{jawltxdoc}}
}
\usedir{tex/latex/uealttr}
\generate{\file{\jobname.cls}{\from{\jobname.dtx}{class}}
}
\usedir{source/latex/uealttr}
\generate{\file{\jobname.ins}{\from{\jobname.dtx}{install}}
}
\nopreamble\nopostamble
\usedir{doc/latex/uealttr}
\generate{\file{README.txt}{\from{\jobname.dtx}{readme}}
}
\endbatchfile
%</install>
%<*readme>
----------------------------------------------------------------
The uealttr package --- A letter class for UEA
Maintained by Joseph Wright
E-mail: joseph.wright@uea.ac.uk
Released under the LaTeX Project Public License v1.3c or later
See http://www.latex-project.org/lppl.txt
----------------------------------------------------------------

The uealttr class is version of the standard LaTeX letter class
customised for use at the University of East Anglia (UEA).  It
is based on the Word template made available by the Publications
Office.  Although aimed at UEA, the class is readily adapted to
other organisations.
%</readme>
%<*ignore>
\fi
% Will Robertson's trick
\immediate\write18{makeindex -s gind.ist -o \jobname.ind  \jobname.idx}
\immediate\write18{makeindex -s gglo.ist -o \jobname.gls  \jobname.glo}
%</ignore>
%<*driver>
\PassOptionsToClass{a4paper}{article}
\documentclass[german,english,UKenglish]{ltxdoc}
\EnableCrossrefs
\CodelineIndex
\RecordChanges
%\OnlyDescription
\usepackage{jawltxdoc}
\begin{document}
  \DocInput{\jobname.dtx}
\end{document}
%</driver>
% \fi
%
%\DoNotIndex{\&,\@date,\@empty,\\,\AddToShipoutPicture,\baselineskip}
%\DoNotIndex{\begin,\ClassInfo,\ClassWarning,\DeclareBoolOption}
%\DoNotIndex{\DeclareComplementaryOption,\def,\else,\end,\faculty}
%\DoNotIndex{\familydefault,\fi,\footnotesize,\fromaddress,\fromname}
%\DoNotIndex{\fromsig,\fromsignemail,\hspace,\IfFileExists,\ifpdf}
%\DoNotIndex{\ifx,\ignorespaces,\includegraphics,\InputIfFileExists}
%\DoNotIndex{\item,\itemindent,\leftmargin,\listparindent,\LoadClass}
%\DoNotIndex{\MessageBreak,\NeedsTeXFormat,\newcommand}
%\DoNotIndex{\newenvironment,\nobreak,\noindent,\overfullrule,\par}
%\DoNotIndex{\parindent,\parsep,\parskip,\ProcessKeyvalOptions}
%\DoNotIndex{\ProvidesClass,\put,\raggedright,\renewcomman}
%\DoNotIndex{\RequirePackage,\rightmargin,\setkeys,\setlength}
%\DoNotIndex{\SetupKeyvalOptions,\sfdefault,\stopbreaks,\strut}
%\DoNotIndex{\textbf,\thispagestyl,\toaddress,\toname,\topsep}
%\DoNotIndex{\unitlength,\vfill,\vspace,\z@,\MakeUppercase}
%\DoNotIndex{\renewcommand,\thispagestyle,\expandafter,\bfseries}
%
%\CheckSum{356}
%
% \CharacterTable
%  {Upper-case    \A\B\C\D\E\F\G\H\I\J\K\L\M\N\O\P\Q\R\S\T\U\V\W\X\Y\Z
%   Lower-case    \a\b\c\d\e\f\g\h\i\j\k\l\m\n\o\p\q\r\s\t\u\v\w\x\y\z
%   Digits        \0\1\2\3\4\5\6\7\8\9
%   Exclamation   \!     Double quote  \"     Hash (number) \#
%   Dollar        \$     Percent       \%     Ampersand     \&
%   Acute accent  \'     Left paren    \(     Right paren   \)
%   Asterisk      \*     Plus          \+     Comma         \,
%   Minus         \-     Point         \.     Solidus       \/
%   Colon         \:     Semicolon     \;     Less than     \<
%   Equals        \=     Greater than  \>     Question mark \?
%   Commercial at \@     Left bracket  \[     Backslash     \\
%   Right bracket \]     Circumflex    \^     Underscore    \_
%   Grave accent  \`     Left brace    \{     Vertical bar  \|
%   Right brace   \}     Tilde         \~}
%
%\def\GetSVNId$#1: #2.#3 #4 #5-#6-#7 #8 #9${%
%  \def\fileversion{v1.0a}%
%  \def\filedate{#5/#6/#7}}
%
%\GetSVNId $Id: uealttr.dtx 14 2008-07-23 22:03:33Z joseph $
%
%\changes{v1.0}{2008/07/21}{First public release}
%
%\setkeys{lst}{language=[LaTeX]{TeX},moretexcs={name,faculty,address,
%  department,email,phone,fax,web,position,closing,opening,logo,
%  subject,RequirePackage}}
%
%\title{\currpkg\ --- A letter class for UEA^^A
%  \thanks{This file describes version \fileversion, last revised
%    \filedate.}}
%\author{Joseph Wright^^A
%  \thanks{E-mail: joseph.wright@uea.ac.uk}}
%\date{Released \filedate}
%
%\maketitle
%
%\begin{abstract}
% The \currpkg class is version of the standard \LaTeX\ letter class
% customised for use at the University of East Anglia (UEA).  It is
% based on the \href
% {http://www1.uea.ac.uk/cm/home/services/units/mac/comm/publicationsoffice/Templates}
% {Word template} made available by the Publications Office. Although
% aimed at UEA, the class is readily adapted to other organisations.
%\end{abstract}
%
%\begin{multicols}{2}
%  \tableofcontents
%\end{multicols}
%
%\section{Introduction}
% The \currpkg class is based on the standard \LaTeX\ class
% \pkg{letter}.  It therefore inherits all of the normal macros from
% the parent: \cs{name}, \cs{opening}, \cs{closing}, \etc.  However,
% the class follows the current guidelines given by UEA for official
% letters. This makes use of a number of additional data macros, and
% also allows ready customisation.  It also makes layout changes to
% include a logo and address information.
%
%\section{Using the class}
%
%\DescribeOption{draft}
%\DescribeOption{final}
% The class is loaded in the usual way, as the argument to
% \cs{documentclass}. The package recognises the \opt{draft} option,
% which will result in the inclusion of thick black bars to show
% overfull boxes.  Any graphics will still be included, as
% \pkg{graphicx} is loaded with the \opt{final} option.
%
%\DescribeMacro{\logo}
% To allow setting up of a graphical logo, the \cs{logo} macro is
% provided by the package.  This is used to set the name  of the file
% containing the logo.  To allow use both with \LaTeX\ and
% pdf\LaTeX\, this macro should not include the file extension. Like
% other \pkg{letter} macros, \cs{logo} takes a single argument.
%\begin{LaTeXexample}[noexample]
%  \logo{uealogo}
%\end{LaTeXexample}
% will therefore cause the class to look for \file{uealogo.eps} if
% compilation uses \LaTeX, or \file{uealogo.pdf} if using pdf\LaTeX.
% The default setting of \cs{logo} is \opt{uealogo}.
%
% For UEA users, the official logo is available as a \ext{eps} file
% from the \href
% {http://www1.uea.ac.uk/cm/home/services/units/mac/comm/publicationsoffice/Logos}
% {Publications Office}. The file can be converted to a \ext{pdf}
% using \pkg{epstopdf}.  Doing this and saving both files to the
% \TeX\ path will allow compilation with either \LaTeX\ or pdf\LaTeX.
%
%\DescribeOption{logo}
% The class is designed so that the first page printed always
% contains space for the logo.  Second and subsequent pages are
% adjusted so that more of the paper is used for printing and the
% logo is not required. The option \opt{logo} governs whether
% the class attempts to print the logo, or simply reserves the space.
% The option takes the values \opt{true} and \opt{false}, using the
% key--value method.  To prevent printing the logo, the class is
% loaded as follows.
%\begin{LaTeXexample}[noexample]
%  \documentclass[logo=false]{uealttr}
%\end{LaTeXexample}
% Note that by default, the class prints the logo (\ie as if
% \opt{logo=true} had been given).
%
%\DescribeOption{personal}
%\DescribeOption{confidential}
% The \opt{personal} and \opt{confidential} options are provided.
% These take Boolean (true/false) values using key--value syntax,
% but can also be given alone.  Thus
%\begin{LaTeXexample}[noexample]
%  \documentclass[personal]{uealttr}
%\end{LaTeXexample}
% and
%\begin{LaTeXexample}[noexample]
%  \documentclass[personal=true]{uealttr}
%\end{LaTeXexample}
% act in the same way.  The two options will include ``PERSONAL'',
% ``CONFIDENTIAL'' or ``PERSONAL \& CONFIDENTIAL'' in the address
% area, if required.
%
%\DescribeMacro{\subject}
%\changes{v1.0a}{2008/07/23}{Altered \cs{subject} macro to alter
%  style used by the Registry}
%\DescribeMacro{\faculty}
%\DescribeMacro{\department}
% A number of pieces of data can be gathered by the standard
% \pkg{letter} class, in macros such as \cs{name}, \cs{address},
% \etc. The \currpkg package adds a number of macros to this list,
% all of which should be given before \cs{opening}. The \cs{subject}
% macro is used to place a subject line in the output. The
% \cs{faculty} and \cs{department} macros include the obvious
% information into the output file, before the contents of
% \cs{address}.
%
%\DescribeMacro{\email}
%\DescribeMacro{\phone}
%\DescribeMacro{\fax}
%\DescribeMacro{\web}
% The macros \cs{email}, \cs{phone} and \cs{fax} are used to include
% general contact details underneath the address area.  In the same
% way, \cs{web} includes a website in the same part of the letter.
% This information will often be general departmental contact
% details.
%\DescribeMacro{\position}
%\DescribeMacro{\signemail}
%\DescribeMacro{\signphone}
% In contrast, \cs{position}, \cs{signemail} and \cs{signphone} add
% information under the signature.  Thus these are intended to relate
% to the person signing the letter.  Notice that the name for the
% signature is taken from \cs{signature} if available, otherwise the
% \cs{name} macro is used.
%
%\section{A demonstration letter}
%
% A simple letter, with all of the data directly in the source, might
% read as follows.
%\begin{LaTeXexample}[noexample]
%  \documentclass[english,UKenglish]{uealttr}
%  \usepackage[final]{microtype}
%  \usepackage{babel}
%  \name{Joseph Wright}
%  \faculty{Faculty of Science}
%  \department{School of Chemical Sciences and Pharmacy}
%  \address{
%    University of East Anglia \\
%    Norwich NR4 7TJ \\
%    United Kingdom}
%  \email{joseph.wright@uea.ac.uk}
%  \phone{+44 (0)1603 591680}
%  \fax{+44 (0)1603 592044}
%  \web{www.uea.ac.uk}
%  \position{Senior Research Associate}
%  \begin{document}
%  \begin{letter}
%  {Mr.~A.N.~Other \\ Some Company \\ Some Street \\ Sometown}
%  \subject{A demonstration letter}
%  \opening{Dear Mr.~Other,}
%
%  This is a rather boring letter, which simply shows how to use
%  the class file.
%
%  \closing{Yours faithfully,}
%  \end{letter}
%  \end{document}
%\end{LaTeXexample}
%
% To make configuration easier, the class will attempt to load a
% configuration file \file{uealttr.cfg}.  This can be used to set up
% repeated data. This can also contain other instructions for \LaTeX.
%  For example, to include the standard data above in every letter,
% the class author uses a configuration file reading
%\begin{LaTeXexample}[noexample]
%  \name{Joseph Wright}
%  \faculty{Faculty of Science}
%  \department{School of Chemical Sciences and Pharmacy}
%  \address{
%    University of East Anglia \\
%    Norwich NR4 7TJ \\
%    United Kingdom}
%  \email{joseph.wright@uea.ac.uk}
%  \phone{+44 (0)1603 591680}
%  \fax{+44 (0)1603 592044}
%  \web{www.uea.ac.uk}
%  \position{Senior Research Associate}
%  \RequirePackage{microtype}
%  \RequirePackage{babel}
%\end{LaTeXexample}
%
% The letter can then be reduced to.
%\begin{LaTeXexample}[noexample]
%  \documentclass[english,UKenglish]{uealttr}
%  \begin{document}
%  \begin{letter}
%  {Mr.~A.N.~Other \\ Some Company \\ Some Street \\ Sometown}
%  \subject{A demonstration letter}
%  \opening{Dear Mr.~Other,}
%
%  This is a rather boring letter, which simply shows how to use
%  the class file.
%
%  \closing{Yours faithfully,}
%  \end{letter}
%  \end{document}
%\end{LaTeXexample}
%
%\StopEventually{%
%  \PrintChanges
%  \PrintIndex}
%
%\iffalse
%<*class>
%\fi
%
%\section{The code}
%\begin{macro}{\uea@id}
% The package starts with the usual identification.
%    \begin{macrocode}
\NeedsTeXFormat{LaTeX2e}
\def\uea@id$#1: #2.#3 #4 #5-#6-#7 #8 #9${#5/#6/#7}
\ProvidesClass{uealttr}
  [\uea@id$Id: uealttr.dtx 14 2008-07-23 22:03:33Z joseph $
   v1.0a A letter class for UEA]
%    \end{macrocode}
%\end{macro}
% The standard support packages are loaded.
%    \begin{macrocode}
\LoadClass[10pt,a4paper]{letter}
\RequirePackage[T1]{fontenc}
\RequirePackage[final]{graphicx}
\RequirePackage[parfill]{parskip}
\RequirePackage{helvet,eso-pic,ifpdf,kvoptions}
\RequirePackage[
  hmargin=30mm,
  vmargin=25mm,
  dvips]{geometry}
%    \end{macrocode}
%\begin{macro}{\ifuea@personal}
%\begin{macro}{\ifuea@confidential}
%\begin{macro}{\ifuea@logo}
%\begin{macro}{\ifuea@draft}
% The single package option is declared.
%    \begin{macrocode}
\SetupKeyvalOptions{
  family = uea,
  prefix = uea@}
\DeclareBoolOption{personal}
\DeclareBoolOption{confidential}
\DeclareBoolOption{logo}
\DeclareBoolOption{draft}
\DeclareComplementaryOption{final}{draft}
\setkeys{uea}{
  personal = false,
  confidential = false,
  logo = true,
  draft = false}
\ProcessKeyvalOptions{uea}
\ifuea@draft
  \setlength\overfullrule{5pt}
\else
  \setlength\overfullrule{0pt}
\fi
%    \end{macrocode}
%\end{macro}
%\end{macro}
%\end{macro}
%\end{macro}
%\begin{macro}{\uea@pandc}
% For personal and confidential letters, some text is set up here.
%    \begin{macrocode}
\newcommand*{\uea@pandc}{}
\ifuea@personal
  \renewcommand*{\uea@pandc}{PERSONAL}
  \ifuea@confidential
    \renewcommand*{\uea@pandc}{PERSONAL \& CONFIDENTIAL}
  \fi
\else
  \ifuea@confidential
    \renewcommand*{\uea@pandc}{CONFIDENTIAL}
  \fi
\fi
%    \end{macrocode}
%\end{macro}
% The UEA style is to use Helvetica for all text.
%    \begin{macrocode}
\renewcommand{\familydefault}{\sfdefault}
%    \end{macrocode}
%\begin{macro}{\subject}
%\begin{macro}{\lettersubject}
% The subject of the letter is set up.
%    \begin{macrocode}
\newcommand*{\subject}[1]{\def\lettersubject{#1}}
\subject{}
%    \end{macrocode}
%\end{macro}
%\end{macro}
%\begin{macro}{\faculty}
%\begin{macro}{\fromfaculty}
%\begin{macro}{\department}
%\begin{macro}{\fromdept}
% A number of macros are used to store the various pieces of data
% used in the address block.  First the faculty and department.
%    \begin{macrocode}
\newcommand*{\faculty}[1]{\def\fromfaculty{#1}}
\newcommand*{\department}[1]{\def\fromdept{#1}}
\faculty{}
\department{}
%    \end{macrocode}
%\end{macro}
%\end{macro}
%\end{macro}
%\end{macro}
%\begin{macro}{\email}
%\begin{macro}{\fromemail}
%\begin{macro}{\phone}
%\begin{macro}{\fromphone}
%\begin{macro}{\fax}
%\begin{macro}{\fromfax}
%\begin{macro}{\web}
%\begin{macro}{\fromweb}
% Next come the contact details for the right-hand area.
%    \begin{macrocode}
\newcommand*{\email}[1]{\def\fromemail{#1}}
\newcommand*{\phone}[1]{\def\fromphone{#1}}
\newcommand*{\fax}[1]{\def\fromfax{#1}}
\newcommand*{\web}[1]{\def\fromweb{#1}}
\email{}
\phone{}
\fax{}
\web{}
%    \end{macrocode}
%\end{macro}
%\end{macro}
%\end{macro}
%\end{macro}
%\end{macro}
%\end{macro}
%\end{macro}
%\end{macro}
%\begin{macro}{\position}
%\begin{macro}{\fromposition}
%\begin{macro}{\signphone}
%\begin{macro}{\fromsignphone}
%\begin{macro}{\signemail}
%\begin{macro}{\fromsignemail}
% Finally, some details added after the signature.
%    \begin{macrocode}
\newcommand*{\position}[1]{\def\fromposition{#1}}
\newcommand*{\signphone}[1]{\def\fromsignphone{#1}}
\newcommand*{\signemail}[1]{\def\fromsignemail{#1}}
\position{}
\signphone{}
\signemail{}
%    \end{macrocode}
%\end{macro}
%\end{macro}
%\end{macro}
%\end{macro}
%\end{macro}
%\end{macro}
%\begin{macro}{\logo}
%\begin{macro}{\fromlogo}
% Any local configuration is loaded, and options are processed.
%    \begin{macrocode}
\newcommand*{\logo}[1]{\def\fromlogo{#1}}
\logo{uealogo}
\InputIfFileExists{uealttr.cfg}
  {\ClassInfo{uealttr}{Loaded local configuration file}}
  {}
%    \end{macrocode}
%\end{macro}
%\end{macro}
% If no logo is available, the package will skip trying to position
% it.
%    \begin{macrocode}
\ifpdf
  \IfFileExists{\fromlogo.pdf}
    {}
    {\ifuea@logo
       \ClassWarning{uealttr}
         {Logo file \fromlogo.pdf not found!\MessageBreak
          No logo will be included in output}
     \fi
     \uea@logofalse}
\else
  \IfFileExists{\fromlogo.eps}
    {}
    {\ifuea@logo
       \ClassWarning{uealttr}
         {Logo file \fromlogo.eps not found!\MessageBreak
          No logo will be included in output}
     \fi
     \uea@logofalse}
\fi
%    \end{macrocode}
%\begin{environment}{wider}
% A trick taken from \pkg{memoir} and the UK FAQ: here, the extra
% width needed is known.
%    \begin{macrocode}
\newenvironment{wider}{%
  \begin{list}{}{%
    \setlength{\topsep}{0pt}%
    \setlength{\leftmargin}{\z@}%
    \setlength{\rightmargin}{-22mm}%
    \setlength{\listparindent}{\parindent}%
    \setlength{\itemindent}{\parindent}%
    \setlength{\parsep}{\parskip}%
  }%
  \item[]}{\end{list}}

%    \end{macrocode}
%\end{environment}
%\begin{macro}{\opening}
% To achieve the correct layout, completely new \cs{opening}
% and \cs{closing} macros are employed.
%    \begin{macrocode}
\renewcommand*{\opening}[1]{%
  \thispagestyle{empty}%
  \vspace*{19mm}%
%    \end{macrocode}
% If the logo has been requested, it is included here.
%    \begin{macrocode}
  \ifuea@logo
    \AddToShipoutPicture*{%
      \setlength{\unitlength}{1mm}%
      \put(147,257){\includegraphics{\fromlogo}}}%
  \fi
%    \end{macrocode}
% The address block is created by using a series of minipages of the
% correct width.  All of this is nested inside a group where the
% margins are made wider.
%    \begin{macrocode}
  \begin{wider}
    \setlength{\leftmargin}{\z@}%
    \setlength{\rightmargin}{22mm}%
    \begin{minipage}[t]{80mm}
      \begin{minipage}[t]{80mm}
        \raggedright
        \ifx\@empty\uea@pandc\else
          \textbf{\uea@pandc} \\*[\baselineskip]
        \fi
        \toname \\
        \toaddress
      \end{minipage}%
%    \end{macrocode}
% A zero-width minipage is used to force the date downward to the
% desired position.
%    \begin{macrocode}
      \begin{minipage}[t]{0mm}
        \vspace*{35mm}%
      \end{minipage}
      \@date
    \end{minipage}%
%    \end{macrocode}
% This minipage generates the whitespace between the two address
% blocks.
%    \begin{macrocode}
    \begin{minipage}[t]{42mm}
      \hspace{42mm}%
      \vspace*{55mm}%
    \end{minipage}%
%    \end{macrocode}
% For the address block, a series of checks are made so that no empty
% lines are ended with |\\|.
%    \begin{macrocode}
    \begin{minipage}[t]{50mm}
      \footnotesize
      \raggedright
      \ifx\@empty\fromfaculty
        \ifx\@empty\fromdept\else
          \ignorespaces\fromdept \\*[\baselineskip]
        \fi
      \else
        \ifx\@empty\fromdept
          \textbf{\ignorespaces\fromfaculty} \\*[\baselineskip]
        \else
          \textbf{\ignorespaces\fromfaculty} \\
          \ignorespaces\fromdept \\*[\baselineskip]
        \fi
      \fi
      \ifx\@empty\fromaddress\else
        \ignorespaces\fromaddress \\*[\baselineskip]
      \fi
      \ifx\@empty\fromemail\else
        Email: \ignorespaces\fromemail \\
      \fi
      \ifx\@empty\fromphone\else
        Tel: \ignorespaces\fromphone \\
      \fi
      \ifx\@empty\fromfax\else
        Fax: \ignorespaces\fromfax \\
      \fi
      \ifx\@empty\fromweb\else
        \ignorespaces\fromweb \\
      \fi
    \end{minipage}%
  \end{wider}
  \par\noindent#1\par\nobreak
  {\bfseries
   \ifx\@empty\lettersubject\else
     \expandafter\uea@MakeUppercase\lettersubject \\\@empty
   \fi}}
%    \end{macrocode}
%\end{macro}
%\begin{macro}{\uea@MakeUppercase}
% A support macro is needed to allow multi-line uppercase.
%    \begin{macrocode}
\def\uea@MakeUppercase#1\\#2\@empty{%
  \MakeUppercase{#1}%
  \ifx\@empty#2\@empty\else
    \\ \expandafter\uea@MakeUppercase#2\@empty
  \fi}
%    \end{macrocode}
%\end{macro}
%\begin{macro}{\closing}
%\darg{closing}
% The same tricks are used for the closing.
%    \begin{macrocode}
\renewcommand{\closing}[1]{%
  \par\nobreak\vspace{\parskip}%
  \stopbreaks
  \noindent
  \ignorespaces #1\\[24mm]
  \ifx\@empty\fromsig
    \fromname \\
  \else
    \fromsig \\
  \fi
  \ifx\@empty\fromposition\else
    \fromposition \\
  \fi
  \ifx\@empty\fromsignphone\else
    \fromsignphone \\
  \fi
  \ifx\@empty\fromsignemail\else
    \fromsignemail \\
  \fi
  \strut
  \par}
%    \end{macrocode}
%\end{macro}
%\begin{macro}{\stopletter}
% The \cs{stopletter} macro is used to push the contents up the page,
% in contrast to normal \LaTeX\ behaviour.
%    \begin{macrocode}
\renewcommand*{\stopletter}{\vfill}
%    \end{macrocode}
%\end{macro}
%
%\Finale
%\iffalse
%</class>
%<*jawltxdoc>
\NeedsTeXFormat{LaTeX2e}
\ProvidesPackage{jawltxdoc}
\usepackage[T1]{fontenc}
\usepackage{lmodern}
\usepackage[final]{listings,graphicx,microtype}
\usepackage[scaled=0.95]{helvet}
\usepackage[version=3]{mhchem}
\usepackage[osf]{mathpazo}
\usepackage{booktabs,array,url,courier,xspace,varioref}
\usepackage{upgreek,ifpdf,float,caption,longtable,babel}
\begingroup
  \@ifundefined{eTeXversion}
    {\aftergroup\@gobble}
    {\aftergroup\@firstofone}
\endgroup
{\usepackage{etoolbox}}
\floatstyle{plaintop}
\restylefloat{table}
\labelformat{figure}{\figurename~#1}
\labelformat{table}{\tablename~#1}
\ifpdf
  \usepackage{embedfile}
  \embedfile[%
    stringmethod=escape,%
    mimetype=plain/text,%
    desc={LaTeX docstrip source archive for package `\jobname'}%
    ]{\jobname.dtx}
\fi
\IfFileExists{\jobname.sty}
  {\usepackage{\jobname}}{}
\usepackage[numbered]{hypdoc}
\setcounter{IndexColumns}{2}
\newlength\LaTeXwidth
\newlength\LaTeXoutdent
\newlength\LaTeXgap
\setlength\LaTeXgap{1em}
\setlength\LaTeXoutdent{-0.15\textwidth}
\newbox\lst@samplebox
\edef\LaTeXexamplefile{\jobname.tmp}
\lst@RequireAspects{writefile}
\lstnewenvironment{LaTeXexample}[1][example]{%
  \global\let\lst@intname\@empty
  \ifcsname LaTeXcode#1\endcsname
    \expandafter\let\expandafter\LaTeXcode
      \csname LaTeXcode#1\endcsname
    \expandafter\let\expandafter\LaTeXcodeend
      \csname LaTeXcode#1end\endcsname
  \else
    \PackageError{jawltxdoc}
      {Undefined example type `#1'}
      \@ehd
    \let\LaTeXcode\relax
    \let\LaTeXcodeend\relax
  \fi
  \LaTeXcode}
  {\lst@EndWriteFile
   \LaTeXcodeend}
\newcommand*{\LaTeXcodeexample}{%
  \setbox\lst@samplebox=\hbox\bgroup
  \LaTeXcodefloat}
\let\LaTeXcoderesultonly\LaTeXcodeexample
\newcommand*{\LaTeXcodeexampleend}{%
  \egroup
  \setlength\LaTeXwidth{\wd\lst@samplebox}%
  \begin{list}{}{%
    \setlength\itemindent{0pt}
    \setlength\leftmargin\LaTeXoutdent
    \setlength\rightmargin{0pt}}%
    \item
      \setlength\LaTeXoutdent{-0.15\textwidth}
      \begin{minipage}[c]{%
        \textwidth-\LaTeXwidth-\LaTeXoutdent-\LaTeXgap}
        \LaTeXcodefloatend
      \end{minipage}%
      \hfill
      \begin{minipage}[c]{\LaTeXwidth}%
        \hbox to\linewidth{\box\lst@samplebox\hss}%
      \end{minipage}%
  \end{list}}
\newcommand*{\LaTeXcodefloat}{%
  \setkeys{lst}{tabsize=4,gobble=3,breakindent=0pt,
    basicstyle=\small\ttfamily,basewidth=0.51em,
    keywordstyle=\color{blue}}%
  \lst@BeginAlsoWriteFile{\LaTeXexamplefile}}
\let\LaTeXcodenoexample\LaTeXcodefloat
\let\LaTeXcodenoexampleend\@empty
\newcommand*{\LaTeXcodefloatend}{%
  \MakePercentComment\catcode`\^^M=10\relax
  \small
  {\setkeys{lst}{SelectCharTable=\lst@ReplaceInput{\^\^I}%
    {\lst@ProcessTabulator}}%
    \leavevmode \input{\LaTeXexamplefile}}%
  \MakePercentIgnore}
\newcommand*{\LaTeXcoderesultonlyend}{\egroup\LaTeXcodefloatend}
\lstnewenvironment{BibTeXexample}{%
  \global\let\lst@intname\@empty
  \setbox\lst@samplebox=\hbox\bgroup
  \setkeys{lst}{tabsize=4,gobble=3,breakindent=0pt,
    basicstyle=\small\ttfamily,basewidth=0.51em,
    keywordstyle=\color{black}}
  \lst@BeginAlsoWriteFile{\LaTeXexamplefile}}
 {\lst@EndWriteFile
   \LaTeXcodeexampleend}
\newcommand*{\DescribeOption}{%
  \leavevmode\@bsphack\begingroup\MakePrivateLetters
  \Describe@Option}
\newcommand*{\Describe@Option}[1]{\endgroup
              \marginpar{\raggedleft\PrintDescribeEnv{#1}}%
              \SpecialOptionIndex{#1}\@esphack\ignorespaces}
\newcommand*{\SpecialOptionIndex}[1]{\@bsphack
    \index{#1\actualchar{\protect\ttfamily#1}
           (option)\encapchar usage}%
    \index{options:\levelchar#1\actualchar{\protect\ttfamily#1}%
      \encapchar usage}\@esphack}
\newcommand*{\indexopt}[1]{\DescribeOption{#1}\opt{#1}}
\newcommand*{\DescribeOptionInfo}[2]{%
  \DescribeOption{#1}%
  \opt{#1=\meta{#2}}\xspace}
\newcommand*{\ofixarg}[1]{%
  {\ttfamily[}%
  \ifmmode \expandafter \nfss@text \fi
  {%
    \meta@font@select
    \edef\meta@hyphen@restore{%
      \hyphenchar\the\font\the\hyphenchar\font}%
    \hyphenchar\font\m@ne
    \language\l@nohyphenation
    #1\/%
    \meta@hyphen@restore
    }%
    {\ttfamily]}}
\newcommand*{\pkg}[1]{\textsf{#1}}
\newcommand*{\currpkg}{\pkg{\jobname}\xspace}
\newcommand*{\opt}[1]{\texttt{#1}}
\newcommand*{\defaultopt}[1]{\opt{\textbf{#1}}}
\newcommand*{\file}[1]{\texttt{#1}}
\newcommand*{\ext}[1]{\file{.#1}}
\newcommand*{\latin}[1]{\emph{#1}}
\newcommand*{\etc}{%
  \@ifnextchar.
    {\latin{etc}}
    {\latin{etc}.\xspace}}
\newcommand*{\eg}{%
  \@ifnextchar.
    {\latin{e.g}}
    {\latin{e.g}.\xspace}}
\newcommand*{\ie}{%
  \@ifnextchar.
    {\latin{i.e}}
    {\latin{i.e}.\xspace}}
\newcommand*{\etal}{%
  \@ifnextchar.
    {\latin{et~al.}}
    {\latin{et~al}.\xspace}}
\newcommand*{\AMS}{{\protect\usefont{OMS}{cmsy}{m}{n}%
  A\kern-.1667em\lower.5ex\hbox{M}\kern-.125emS}}
\providecommand*{\eTeX}{\ensuremath{\varepsilon}-\TeX}
\DeclareRobustCommand*{\XeTeX}
  {X\kern-.125em\lower.5ex\hbox{\reflectbox{E}}\kern-.1667em\TeX}
\providecommand*{\CTAN}{\textsc{ctan}}
\@ifpackageloaded{etoolbox}
  {\patchcmd{\@addmarginpar}
    {\@latex@warning@no@line {Marginpar on page \thepage\space moved}}
    {\relax}{}{}}
  {}
\newcounter{argument}
\g@addto@macro\endmacro{\setcounter{argument}{0}}
\newcommand*\darg[1]{%
  \stepcounter{argument}%
  {\ttfamily\char`\#\theargument~:~}#1\par\noindent\ignorespaces}
\newcommand*\doarg[1]{%
  \stepcounter{argument}%
  {\ttfamily\makebox[0pt][r]{[}%
   \char`\#\theargument]:~}#1\par\noindent\ignorespaces}
%</jawltxdoc>
%\fi

%
% Documentation:
%    (a) Without write18 enabled:
%          pdflatex uealttr.dtx
%          makeindex -s gind.ist uealttr.idx
%          makeindex -s gglo.ist -o uealttr.gls uealttr.glo
%          pdflatex uealttr.dtx
%          pdflatex uealttr.dtx
%    (b) With write18 enabled:
%          pdflatex uealttr.dtx
%          pdflatex uealttr.dtx
%          pdflatex uealttr.dtx
%
% Installation:
%     Copy uealttr.sty to a location searched by TeX, and if required
%     by your TeX installation, run the appropriate command to build
%     a hash of files (texhash, initexmf --update-fndb, etc.)
%
% Note:
%     The jawltxdoc.sty file is not needed for installation, only for
%     building the documentation; it may be deleted after producing
%     the documentation (if necessary).
%
%<*ignore>
% This is all taken verbatim from Heiko Oberdiek's packages
\begingroup
  \def\x{LaTeX2e}%
\expandafter\endgroup
\ifcase 0\ifx\install y1\fi\expandafter
         \ifx\csname processbatchFile\endcsname\relax\else1\fi
         \ifx\fmtname\x\else 1\fi\relax
\else\csname fi\endcsname
%</ignore>
%<*install>
\input docstrip.tex
\keepsilent
\askforoverwritefalse
\preamble
 ----------------------------------------------------------------
 The uealttr package --- A letter class for UEA
 Maintained by Joseph Wright
 E-mail: joseph.wright@uea.ac.uk
 Released under the LaTeX Project Public License v1.3c or later
 See http://www.latex-project.org/lppl.txt
 ----------------------------------------------------------------

\endpreamble
\Msg{Generating uealttr files:}
\generate{\file{jawltxdoc.sty}{\from{\jobname.dtx}{jawltxdoc}}
}
\usedir{tex/latex/uealttr}
\generate{\file{\jobname.cls}{\from{\jobname.dtx}{class}}
}
\usedir{source/latex/uealttr}
\generate{\file{\jobname.ins}{\from{\jobname.dtx}{install}}
}
\nopreamble\nopostamble
\usedir{doc/latex/uealttr}
\generate{\file{README.txt}{\from{\jobname.dtx}{readme}}
}
\endbatchfile
%</install>
%<*readme>
----------------------------------------------------------------
The uealttr package --- A letter class for UEA
Maintained by Joseph Wright
E-mail: joseph.wright@uea.ac.uk
Released under the LaTeX Project Public License v1.3c or later
See http://www.latex-project.org/lppl.txt
----------------------------------------------------------------

The uealttr class is version of the standard LaTeX letter class
customised for use at the University of East Anglia (UEA).  It
is based on the Word template made available by the Publications
Office.  Although aimed at UEA, the class is readily adapted to
other organisations.
%</readme>
%<*ignore>
\fi
% Will Robertson's trick
\immediate\write18{makeindex -s gind.ist -o \jobname.ind  \jobname.idx}
\immediate\write18{makeindex -s gglo.ist -o \jobname.gls  \jobname.glo}
%</ignore>
%<*driver>
\PassOptionsToClass{a4paper}{article}
\documentclass[german,english,UKenglish]{ltxdoc}
\EnableCrossrefs
\CodelineIndex
\RecordChanges
%\OnlyDescription
\usepackage{jawltxdoc}
\begin{document}
  \DocInput{\jobname.dtx}
\end{document}
%</driver>
% \fi
%
%\DoNotIndex{\&,\@date,\@empty,\\,\AddToShipoutPicture,\baselineskip}
%\DoNotIndex{\begin,\ClassInfo,\ClassWarning,\DeclareBoolOption}
%\DoNotIndex{\DeclareComplementaryOption,\def,\else,\end,\faculty}
%\DoNotIndex{\familydefault,\fi,\footnotesize,\fromaddress,\fromname}
%\DoNotIndex{\fromsig,\fromsignemail,\hspace,\IfFileExists,\ifpdf}
%\DoNotIndex{\ifx,\ignorespaces,\includegraphics,\InputIfFileExists}
%\DoNotIndex{\item,\itemindent,\leftmargin,\listparindent,\LoadClass}
%\DoNotIndex{\MessageBreak,\NeedsTeXFormat,\newcommand}
%\DoNotIndex{\newenvironment,\nobreak,\noindent,\overfullrule,\par}
%\DoNotIndex{\parindent,\parsep,\parskip,\ProcessKeyvalOptions}
%\DoNotIndex{\ProvidesClass,\put,\raggedright,\renewcomman}
%\DoNotIndex{\RequirePackage,\rightmargin,\setkeys,\setlength}
%\DoNotIndex{\SetupKeyvalOptions,\sfdefault,\stopbreaks,\strut}
%\DoNotIndex{\textbf,\thispagestyl,\toaddress,\toname,\topsep}
%\DoNotIndex{\unitlength,\vfill,\vspace,\z@,\MakeUppercase}
%\DoNotIndex{\renewcommand,\thispagestyle,\expandafter,\bfseries}
%
%\CheckSum{356}
%
% \CharacterTable
%  {Upper-case    \A\B\C\D\E\F\G\H\I\J\K\L\M\N\O\P\Q\R\S\T\U\V\W\X\Y\Z
%   Lower-case    \a\b\c\d\e\f\g\h\i\j\k\l\m\n\o\p\q\r\s\t\u\v\w\x\y\z
%   Digits        \0\1\2\3\4\5\6\7\8\9
%   Exclamation   \!     Double quote  \"     Hash (number) \#
%   Dollar        \$     Percent       \%     Ampersand     \&
%   Acute accent  \'     Left paren    \(     Right paren   \)
%   Asterisk      \*     Plus          \+     Comma         \,
%   Minus         \-     Point         \.     Solidus       \/
%   Colon         \:     Semicolon     \;     Less than     \<
%   Equals        \=     Greater than  \>     Question mark \?
%   Commercial at \@     Left bracket  \[     Backslash     \\
%   Right bracket \]     Circumflex    \^     Underscore    \_
%   Grave accent  \`     Left brace    \{     Vertical bar  \|
%   Right brace   \}     Tilde         \~}
%
%\def\GetSVNId$#1: #2.#3 #4 #5-#6-#7 #8 #9${%
%  \def\fileversion{v1.0a}%
%  \def\filedate{#5/#6/#7}}
%
%\GetSVNId $Id: uealttr.dtx 14 2008-07-23 22:03:33Z joseph $
%
%\changes{v1.0}{2008/07/21}{First public release}
%
%\setkeys{lst}{language=[LaTeX]{TeX},moretexcs={name,faculty,address,
%  department,email,phone,fax,web,position,closing,opening,logo,
%  subject,RequirePackage}}
%
%\title{\currpkg\ --- A letter class for UEA^^A
%  \thanks{This file describes version \fileversion, last revised
%    \filedate.}}
%\author{Joseph Wright^^A
%  \thanks{E-mail: joseph.wright@uea.ac.uk}}
%\date{Released \filedate}
%
%\maketitle
%
%\begin{abstract}
% The \currpkg class is version of the standard \LaTeX\ letter class
% customised for use at the University of East Anglia (UEA).  It is
% based on the \href
% {http://www1.uea.ac.uk/cm/home/services/units/mac/comm/publicationsoffice/Templates}
% {Word template} made available by the Publications Office. Although
% aimed at UEA, the class is readily adapted to other organisations.
%\end{abstract}
%
%\begin{multicols}{2}
%  \tableofcontents
%\end{multicols}
%
%\section{Introduction}
% The \currpkg class is based on the standard \LaTeX\ class
% \pkg{letter}.  It therefore inherits all of the normal macros from
% the parent: \cs{name}, \cs{opening}, \cs{closing}, \etc.  However,
% the class follows the current guidelines given by UEA for official
% letters. This makes use of a number of additional data macros, and
% also allows ready customisation.  It also makes layout changes to
% include a logo and address information.
%
%\section{Using the class}
%
%\DescribeOption{draft}
%\DescribeOption{final}
% The class is loaded in the usual way, as the argument to
% \cs{documentclass}. The package recognises the \opt{draft} option,
% which will result in the inclusion of thick black bars to show
% overfull boxes.  Any graphics will still be included, as
% \pkg{graphicx} is loaded with the \opt{final} option.
%
%\DescribeMacro{\logo}
% To allow setting up of a graphical logo, the \cs{logo} macro is
% provided by the package.  This is used to set the name  of the file
% containing the logo.  To allow use both with \LaTeX\ and
% pdf\LaTeX\, this macro should not include the file extension. Like
% other \pkg{letter} macros, \cs{logo} takes a single argument.
%\begin{LaTeXexample}[noexample]
%  \logo{uealogo}
%\end{LaTeXexample}
% will therefore cause the class to look for \file{uealogo.eps} if
% compilation uses \LaTeX, or \file{uealogo.pdf} if using pdf\LaTeX.
% The default setting of \cs{logo} is \opt{uealogo}.
%
% For UEA users, the official logo is available as a \ext{eps} file
% from the \href
% {http://www1.uea.ac.uk/cm/home/services/units/mac/comm/publicationsoffice/Logos}
% {Publications Office}. The file can be converted to a \ext{pdf}
% using \pkg{epstopdf}.  Doing this and saving both files to the
% \TeX\ path will allow compilation with either \LaTeX\ or pdf\LaTeX.
%
%\DescribeOption{logo}
% The class is designed so that the first page printed always
% contains space for the logo.  Second and subsequent pages are
% adjusted so that more of the paper is used for printing and the
% logo is not required. The option \opt{logo} governs whether
% the class attempts to print the logo, or simply reserves the space.
% The option takes the values \opt{true} and \opt{false}, using the
% key--value method.  To prevent printing the logo, the class is
% loaded as follows.
%\begin{LaTeXexample}[noexample]
%  \documentclass[logo=false]{uealttr}
%\end{LaTeXexample}
% Note that by default, the class prints the logo (\ie as if
% \opt{logo=true} had been given).
%
%\DescribeOption{personal}
%\DescribeOption{confidential}
% The \opt{personal} and \opt{confidential} options are provided.
% These take Boolean (true/false) values using key--value syntax,
% but can also be given alone.  Thus
%\begin{LaTeXexample}[noexample]
%  \documentclass[personal]{uealttr}
%\end{LaTeXexample}
% and
%\begin{LaTeXexample}[noexample]
%  \documentclass[personal=true]{uealttr}
%\end{LaTeXexample}
% act in the same way.  The two options will include ``PERSONAL'',
% ``CONFIDENTIAL'' or ``PERSONAL \& CONFIDENTIAL'' in the address
% area, if required.
%
%\DescribeMacro{\subject}
%\changes{v1.0a}{2008/07/23}{Altered \cs{subject} macro to alter
%  style used by the Registry}
%\DescribeMacro{\faculty}
%\DescribeMacro{\department}
% A number of pieces of data can be gathered by the standard
% \pkg{letter} class, in macros such as \cs{name}, \cs{address},
% \etc. The \currpkg package adds a number of macros to this list,
% all of which should be given before \cs{opening}. The \cs{subject}
% macro is used to place a subject line in the output. The
% \cs{faculty} and \cs{department} macros include the obvious
% information into the output file, before the contents of
% \cs{address}.
%
%\DescribeMacro{\email}
%\DescribeMacro{\phone}
%\DescribeMacro{\fax}
%\DescribeMacro{\web}
% The macros \cs{email}, \cs{phone} and \cs{fax} are used to include
% general contact details underneath the address area.  In the same
% way, \cs{web} includes a website in the same part of the letter.
% This information will often be general departmental contact
% details.
%\DescribeMacro{\position}
%\DescribeMacro{\signemail}
%\DescribeMacro{\signphone}
% In contrast, \cs{position}, \cs{signemail} and \cs{signphone} add
% information under the signature.  Thus these are intended to relate
% to the person signing the letter.  Notice that the name for the
% signature is taken from \cs{signature} if available, otherwise the
% \cs{name} macro is used.
%
%\section{A demonstration letter}
%
% A simple letter, with all of the data directly in the source, might
% read as follows.
%\begin{LaTeXexample}[noexample]
%  \documentclass[english,UKenglish]{uealttr}
%  \usepackage[final]{microtype}
%  \usepackage{babel}
%  \name{Joseph Wright}
%  \faculty{Faculty of Science}
%  \department{School of Chemical Sciences and Pharmacy}
%  \address{
%    University of East Anglia \\
%    Norwich NR4 7TJ \\
%    United Kingdom}
%  \email{joseph.wright@uea.ac.uk}
%  \phone{+44 (0)1603 591680}
%  \fax{+44 (0)1603 592044}
%  \web{www.uea.ac.uk}
%  \position{Senior Research Associate}
%  \begin{document}
%  \begin{letter}
%  {Mr.~A.N.~Other \\ Some Company \\ Some Street \\ Sometown}
%  \subject{A demonstration letter}
%  \opening{Dear Mr.~Other,}
%
%  This is a rather boring letter, which simply shows how to use
%  the class file.
%
%  \closing{Yours faithfully,}
%  \end{letter}
%  \end{document}
%\end{LaTeXexample}
%
% To make configuration easier, the class will attempt to load a
% configuration file \file{uealttr.cfg}.  This can be used to set up
% repeated data. This can also contain other instructions for \LaTeX.
%  For example, to include the standard data above in every letter,
% the class author uses a configuration file reading
%\begin{LaTeXexample}[noexample]
%  \name{Joseph Wright}
%  \faculty{Faculty of Science}
%  \department{School of Chemical Sciences and Pharmacy}
%  \address{
%    University of East Anglia \\
%    Norwich NR4 7TJ \\
%    United Kingdom}
%  \email{joseph.wright@uea.ac.uk}
%  \phone{+44 (0)1603 591680}
%  \fax{+44 (0)1603 592044}
%  \web{www.uea.ac.uk}
%  \position{Senior Research Associate}
%  \RequirePackage{microtype}
%  \RequirePackage{babel}
%\end{LaTeXexample}
%
% The letter can then be reduced to.
%\begin{LaTeXexample}[noexample]
%  \documentclass[english,UKenglish]{uealttr}
%  \begin{document}
%  \begin{letter}
%  {Mr.~A.N.~Other \\ Some Company \\ Some Street \\ Sometown}
%  \subject{A demonstration letter}
%  \opening{Dear Mr.~Other,}
%
%  This is a rather boring letter, which simply shows how to use
%  the class file.
%
%  \closing{Yours faithfully,}
%  \end{letter}
%  \end{document}
%\end{LaTeXexample}
%
%\StopEventually{%
%  \PrintChanges
%  \PrintIndex}
%
%\iffalse
%<*class>
%\fi
%
%\section{The code}
%\begin{macro}{\uea@id}
% The package starts with the usual identification.
%    \begin{macrocode}
\NeedsTeXFormat{LaTeX2e}
\def\uea@id$#1: #2.#3 #4 #5-#6-#7 #8 #9${#5/#6/#7}
\ProvidesClass{uealttr}
  [\uea@id$Id: uealttr.dtx 14 2008-07-23 22:03:33Z joseph $
   v1.0a A letter class for UEA]
%    \end{macrocode}
%\end{macro}
% The standard support packages are loaded.
%    \begin{macrocode}
\LoadClass[10pt,a4paper]{letter}
\RequirePackage[T1]{fontenc}
\RequirePackage[final]{graphicx}
\RequirePackage[parfill]{parskip}
\RequirePackage{helvet,eso-pic,ifpdf,kvoptions}
\RequirePackage[
  hmargin=30mm,
  vmargin=25mm,
  dvips]{geometry}
%    \end{macrocode}
%\begin{macro}{\ifuea@personal}
%\begin{macro}{\ifuea@confidential}
%\begin{macro}{\ifuea@logo}
%\begin{macro}{\ifuea@draft}
% The single package option is declared.
%    \begin{macrocode}
\SetupKeyvalOptions{
  family = uea,
  prefix = uea@}
\DeclareBoolOption{personal}
\DeclareBoolOption{confidential}
\DeclareBoolOption{logo}
\DeclareBoolOption{draft}
\DeclareComplementaryOption{final}{draft}
\setkeys{uea}{
  personal = false,
  confidential = false,
  logo = true,
  draft = false}
\ProcessKeyvalOptions{uea}
\ifuea@draft
  \setlength\overfullrule{5pt}
\else
  \setlength\overfullrule{0pt}
\fi
%    \end{macrocode}
%\end{macro}
%\end{macro}
%\end{macro}
%\end{macro}
%\begin{macro}{\uea@pandc}
% For personal and confidential letters, some text is set up here.
%    \begin{macrocode}
\newcommand*{\uea@pandc}{}
\ifuea@personal
  \renewcommand*{\uea@pandc}{PERSONAL}
  \ifuea@confidential
    \renewcommand*{\uea@pandc}{PERSONAL \& CONFIDENTIAL}
  \fi
\else
  \ifuea@confidential
    \renewcommand*{\uea@pandc}{CONFIDENTIAL}
  \fi
\fi
%    \end{macrocode}
%\end{macro}
% The UEA style is to use Helvetica for all text.
%    \begin{macrocode}
\renewcommand{\familydefault}{\sfdefault}
%    \end{macrocode}
%\begin{macro}{\subject}
%\begin{macro}{\lettersubject}
% The subject of the letter is set up.
%    \begin{macrocode}
\newcommand*{\subject}[1]{\def\lettersubject{#1}}
\subject{}
%    \end{macrocode}
%\end{macro}
%\end{macro}
%\begin{macro}{\faculty}
%\begin{macro}{\fromfaculty}
%\begin{macro}{\department}
%\begin{macro}{\fromdept}
% A number of macros are used to store the various pieces of data
% used in the address block.  First the faculty and department.
%    \begin{macrocode}
\newcommand*{\faculty}[1]{\def\fromfaculty{#1}}
\newcommand*{\department}[1]{\def\fromdept{#1}}
\faculty{}
\department{}
%    \end{macrocode}
%\end{macro}
%\end{macro}
%\end{macro}
%\end{macro}
%\begin{macro}{\email}
%\begin{macro}{\fromemail}
%\begin{macro}{\phone}
%\begin{macro}{\fromphone}
%\begin{macro}{\fax}
%\begin{macro}{\fromfax}
%\begin{macro}{\web}
%\begin{macro}{\fromweb}
% Next come the contact details for the right-hand area.
%    \begin{macrocode}
\newcommand*{\email}[1]{\def\fromemail{#1}}
\newcommand*{\phone}[1]{\def\fromphone{#1}}
\newcommand*{\fax}[1]{\def\fromfax{#1}}
\newcommand*{\web}[1]{\def\fromweb{#1}}
\email{}
\phone{}
\fax{}
\web{}
%    \end{macrocode}
%\end{macro}
%\end{macro}
%\end{macro}
%\end{macro}
%\end{macro}
%\end{macro}
%\end{macro}
%\end{macro}
%\begin{macro}{\position}
%\begin{macro}{\fromposition}
%\begin{macro}{\signphone}
%\begin{macro}{\fromsignphone}
%\begin{macro}{\signemail}
%\begin{macro}{\fromsignemail}
% Finally, some details added after the signature.
%    \begin{macrocode}
\newcommand*{\position}[1]{\def\fromposition{#1}}
\newcommand*{\signphone}[1]{\def\fromsignphone{#1}}
\newcommand*{\signemail}[1]{\def\fromsignemail{#1}}
\position{}
\signphone{}
\signemail{}
%    \end{macrocode}
%\end{macro}
%\end{macro}
%\end{macro}
%\end{macro}
%\end{macro}
%\end{macro}
%\begin{macro}{\logo}
%\begin{macro}{\fromlogo}
% Any local configuration is loaded, and options are processed.
%    \begin{macrocode}
\newcommand*{\logo}[1]{\def\fromlogo{#1}}
\logo{uealogo}
\InputIfFileExists{uealttr.cfg}
  {\ClassInfo{uealttr}{Loaded local configuration file}}
  {}
%    \end{macrocode}
%\end{macro}
%\end{macro}
% If no logo is available, the package will skip trying to position
% it.
%    \begin{macrocode}
\ifpdf
  \IfFileExists{\fromlogo.pdf}
    {}
    {\ifuea@logo
       \ClassWarning{uealttr}
         {Logo file \fromlogo.pdf not found!\MessageBreak
          No logo will be included in output}
     \fi
     \uea@logofalse}
\else
  \IfFileExists{\fromlogo.eps}
    {}
    {\ifuea@logo
       \ClassWarning{uealttr}
         {Logo file \fromlogo.eps not found!\MessageBreak
          No logo will be included in output}
     \fi
     \uea@logofalse}
\fi
%    \end{macrocode}
%\begin{environment}{wider}
% A trick taken from \pkg{memoir} and the UK FAQ: here, the extra
% width needed is known.
%    \begin{macrocode}
\newenvironment{wider}{%
  \begin{list}{}{%
    \setlength{\topsep}{0pt}%
    \setlength{\leftmargin}{\z@}%
    \setlength{\rightmargin}{-22mm}%
    \setlength{\listparindent}{\parindent}%
    \setlength{\itemindent}{\parindent}%
    \setlength{\parsep}{\parskip}%
  }%
  \item[]}{\end{list}}

%    \end{macrocode}
%\end{environment}
%\begin{macro}{\opening}
% To achieve the correct layout, completely new \cs{opening}
% and \cs{closing} macros are employed.
%    \begin{macrocode}
\renewcommand*{\opening}[1]{%
  \thispagestyle{empty}%
  \vspace*{19mm}%
%    \end{macrocode}
% If the logo has been requested, it is included here.
%    \begin{macrocode}
  \ifuea@logo
    \AddToShipoutPicture*{%
      \setlength{\unitlength}{1mm}%
      \put(147,257){\includegraphics{\fromlogo}}}%
  \fi
%    \end{macrocode}
% The address block is created by using a series of minipages of the
% correct width.  All of this is nested inside a group where the
% margins are made wider.
%    \begin{macrocode}
  \begin{wider}
    \setlength{\leftmargin}{\z@}%
    \setlength{\rightmargin}{22mm}%
    \begin{minipage}[t]{80mm}
      \begin{minipage}[t]{80mm}
        \raggedright
        \ifx\@empty\uea@pandc\else
          \textbf{\uea@pandc} \\*[\baselineskip]
        \fi
        \toname \\
        \toaddress
      \end{minipage}%
%    \end{macrocode}
% A zero-width minipage is used to force the date downward to the
% desired position.
%    \begin{macrocode}
      \begin{minipage}[t]{0mm}
        \vspace*{35mm}%
      \end{minipage}
      \@date
    \end{minipage}%
%    \end{macrocode}
% This minipage generates the whitespace between the two address
% blocks.
%    \begin{macrocode}
    \begin{minipage}[t]{42mm}
      \hspace{42mm}%
      \vspace*{55mm}%
    \end{minipage}%
%    \end{macrocode}
% For the address block, a series of checks are made so that no empty
% lines are ended with |\\|.
%    \begin{macrocode}
    \begin{minipage}[t]{50mm}
      \footnotesize
      \raggedright
      \ifx\@empty\fromfaculty
        \ifx\@empty\fromdept\else
          \ignorespaces\fromdept \\*[\baselineskip]
        \fi
      \else
        \ifx\@empty\fromdept
          \textbf{\ignorespaces\fromfaculty} \\*[\baselineskip]
        \else
          \textbf{\ignorespaces\fromfaculty} \\
          \ignorespaces\fromdept \\*[\baselineskip]
        \fi
      \fi
      \ifx\@empty\fromaddress\else
        \ignorespaces\fromaddress \\*[\baselineskip]
      \fi
      \ifx\@empty\fromemail\else
        Email: \ignorespaces\fromemail \\
      \fi
      \ifx\@empty\fromphone\else
        Tel: \ignorespaces\fromphone \\
      \fi
      \ifx\@empty\fromfax\else
        Fax: \ignorespaces\fromfax \\
      \fi
      \ifx\@empty\fromweb\else
        \ignorespaces\fromweb \\
      \fi
    \end{minipage}%
  \end{wider}
  \par\noindent#1\par\nobreak
  {\bfseries
   \ifx\@empty\lettersubject\else
     \expandafter\uea@MakeUppercase\lettersubject \\\@empty
   \fi}}
%    \end{macrocode}
%\end{macro}
%\begin{macro}{\uea@MakeUppercase}
% A support macro is needed to allow multi-line uppercase.
%    \begin{macrocode}
\def\uea@MakeUppercase#1\\#2\@empty{%
  \MakeUppercase{#1}%
  \ifx\@empty#2\@empty\else
    \\ \expandafter\uea@MakeUppercase#2\@empty
  \fi}
%    \end{macrocode}
%\end{macro}
%\begin{macro}{\closing}
%\darg{closing}
% The same tricks are used for the closing.
%    \begin{macrocode}
\renewcommand{\closing}[1]{%
  \par\nobreak\vspace{\parskip}%
  \stopbreaks
  \noindent
  \ignorespaces #1\\[24mm]
  \ifx\@empty\fromsig
    \fromname \\
  \else
    \fromsig \\
  \fi
  \ifx\@empty\fromposition\else
    \fromposition \\
  \fi
  \ifx\@empty\fromsignphone\else
    \fromsignphone \\
  \fi
  \ifx\@empty\fromsignemail\else
    \fromsignemail \\
  \fi
  \strut
  \par}
%    \end{macrocode}
%\end{macro}
%\begin{macro}{\stopletter}
% The \cs{stopletter} macro is used to push the contents up the page,
% in contrast to normal \LaTeX\ behaviour.
%    \begin{macrocode}
\renewcommand*{\stopletter}{\vfill}
%    \end{macrocode}
%\end{macro}
%
%\Finale
%\iffalse
%</class>
%<*jawltxdoc>
\NeedsTeXFormat{LaTeX2e}
\ProvidesPackage{jawltxdoc}
\usepackage[T1]{fontenc}
\usepackage{lmodern}
\usepackage[final]{listings,graphicx,microtype}
\usepackage[scaled=0.95]{helvet}
\usepackage[version=3]{mhchem}
\usepackage[osf]{mathpazo}
\usepackage{booktabs,array,url,courier,xspace,varioref}
\usepackage{upgreek,ifpdf,float,caption,longtable,babel}
\begingroup
  \@ifundefined{eTeXversion}
    {\aftergroup\@gobble}
    {\aftergroup\@firstofone}
\endgroup
{\usepackage{etoolbox}}
\floatstyle{plaintop}
\restylefloat{table}
\labelformat{figure}{\figurename~#1}
\labelformat{table}{\tablename~#1}
\ifpdf
  \usepackage{embedfile}
  \embedfile[%
    stringmethod=escape,%
    mimetype=plain/text,%
    desc={LaTeX docstrip source archive for package `\jobname'}%
    ]{\jobname.dtx}
\fi
\IfFileExists{\jobname.sty}
  {\usepackage{\jobname}}{}
\usepackage[numbered]{hypdoc}
\setcounter{IndexColumns}{2}
\newlength\LaTeXwidth
\newlength\LaTeXoutdent
\newlength\LaTeXgap
\setlength\LaTeXgap{1em}
\setlength\LaTeXoutdent{-0.15\textwidth}
\newbox\lst@samplebox
\edef\LaTeXexamplefile{\jobname.tmp}
\lst@RequireAspects{writefile}
\lstnewenvironment{LaTeXexample}[1][example]{%
  \global\let\lst@intname\@empty
  \ifcsname LaTeXcode#1\endcsname
    \expandafter\let\expandafter\LaTeXcode
      \csname LaTeXcode#1\endcsname
    \expandafter\let\expandafter\LaTeXcodeend
      \csname LaTeXcode#1end\endcsname
  \else
    \PackageError{jawltxdoc}
      {Undefined example type `#1'}
      \@ehd
    \let\LaTeXcode\relax
    \let\LaTeXcodeend\relax
  \fi
  \LaTeXcode}
  {\lst@EndWriteFile
   \LaTeXcodeend}
\newcommand*{\LaTeXcodeexample}{%
  \setbox\lst@samplebox=\hbox\bgroup
  \LaTeXcodefloat}
\let\LaTeXcoderesultonly\LaTeXcodeexample
\newcommand*{\LaTeXcodeexampleend}{%
  \egroup
  \setlength\LaTeXwidth{\wd\lst@samplebox}%
  \begin{list}{}{%
    \setlength\itemindent{0pt}
    \setlength\leftmargin\LaTeXoutdent
    \setlength\rightmargin{0pt}}%
    \item
      \setlength\LaTeXoutdent{-0.15\textwidth}
      \begin{minipage}[c]{%
        \textwidth-\LaTeXwidth-\LaTeXoutdent-\LaTeXgap}
        \LaTeXcodefloatend
      \end{minipage}%
      \hfill
      \begin{minipage}[c]{\LaTeXwidth}%
        \hbox to\linewidth{\box\lst@samplebox\hss}%
      \end{minipage}%
  \end{list}}
\newcommand*{\LaTeXcodefloat}{%
  \setkeys{lst}{tabsize=4,gobble=3,breakindent=0pt,
    basicstyle=\small\ttfamily,basewidth=0.51em,
    keywordstyle=\color{blue}}%
  \lst@BeginAlsoWriteFile{\LaTeXexamplefile}}
\let\LaTeXcodenoexample\LaTeXcodefloat
\let\LaTeXcodenoexampleend\@empty
\newcommand*{\LaTeXcodefloatend}{%
  \MakePercentComment\catcode`\^^M=10\relax
  \small
  {\setkeys{lst}{SelectCharTable=\lst@ReplaceInput{\^\^I}%
    {\lst@ProcessTabulator}}%
    \leavevmode \input{\LaTeXexamplefile}}%
  \MakePercentIgnore}
\newcommand*{\LaTeXcoderesultonlyend}{\egroup\LaTeXcodefloatend}
\lstnewenvironment{BibTeXexample}{%
  \global\let\lst@intname\@empty
  \setbox\lst@samplebox=\hbox\bgroup
  \setkeys{lst}{tabsize=4,gobble=3,breakindent=0pt,
    basicstyle=\small\ttfamily,basewidth=0.51em,
    keywordstyle=\color{black}}
  \lst@BeginAlsoWriteFile{\LaTeXexamplefile}}
 {\lst@EndWriteFile
   \LaTeXcodeexampleend}
\newcommand*{\DescribeOption}{%
  \leavevmode\@bsphack\begingroup\MakePrivateLetters
  \Describe@Option}
\newcommand*{\Describe@Option}[1]{\endgroup
              \marginpar{\raggedleft\PrintDescribeEnv{#1}}%
              \SpecialOptionIndex{#1}\@esphack\ignorespaces}
\newcommand*{\SpecialOptionIndex}[1]{\@bsphack
    \index{#1\actualchar{\protect\ttfamily#1}
           (option)\encapchar usage}%
    \index{options:\levelchar#1\actualchar{\protect\ttfamily#1}%
      \encapchar usage}\@esphack}
\newcommand*{\indexopt}[1]{\DescribeOption{#1}\opt{#1}}
\newcommand*{\DescribeOptionInfo}[2]{%
  \DescribeOption{#1}%
  \opt{#1=\meta{#2}}\xspace}
\newcommand*{\ofixarg}[1]{%
  {\ttfamily[}%
  \ifmmode \expandafter \nfss@text \fi
  {%
    \meta@font@select
    \edef\meta@hyphen@restore{%
      \hyphenchar\the\font\the\hyphenchar\font}%
    \hyphenchar\font\m@ne
    \language\l@nohyphenation
    #1\/%
    \meta@hyphen@restore
    }%
    {\ttfamily]}}
\newcommand*{\pkg}[1]{\textsf{#1}}
\newcommand*{\currpkg}{\pkg{\jobname}\xspace}
\newcommand*{\opt}[1]{\texttt{#1}}
\newcommand*{\defaultopt}[1]{\opt{\textbf{#1}}}
\newcommand*{\file}[1]{\texttt{#1}}
\newcommand*{\ext}[1]{\file{.#1}}
\newcommand*{\latin}[1]{\emph{#1}}
\newcommand*{\etc}{%
  \@ifnextchar.
    {\latin{etc}}
    {\latin{etc}.\xspace}}
\newcommand*{\eg}{%
  \@ifnextchar.
    {\latin{e.g}}
    {\latin{e.g}.\xspace}}
\newcommand*{\ie}{%
  \@ifnextchar.
    {\latin{i.e}}
    {\latin{i.e}.\xspace}}
\newcommand*{\etal}{%
  \@ifnextchar.
    {\latin{et~al.}}
    {\latin{et~al}.\xspace}}
\newcommand*{\AMS}{{\protect\usefont{OMS}{cmsy}{m}{n}%
  A\kern-.1667em\lower.5ex\hbox{M}\kern-.125emS}}
\providecommand*{\eTeX}{\ensuremath{\varepsilon}-\TeX}
\DeclareRobustCommand*{\XeTeX}
  {X\kern-.125em\lower.5ex\hbox{\reflectbox{E}}\kern-.1667em\TeX}
\providecommand*{\CTAN}{\textsc{ctan}}
\@ifpackageloaded{etoolbox}
  {\patchcmd{\@addmarginpar}
    {\@latex@warning@no@line {Marginpar on page \thepage\space moved}}
    {\relax}{}{}}
  {}
\newcounter{argument}
\g@addto@macro\endmacro{\setcounter{argument}{0}}
\newcommand*\darg[1]{%
  \stepcounter{argument}%
  {\ttfamily\char`\#\theargument~:~}#1\par\noindent\ignorespaces}
\newcommand*\doarg[1]{%
  \stepcounter{argument}%
  {\ttfamily\makebox[0pt][r]{[}%
   \char`\#\theargument]:~}#1\par\noindent\ignorespaces}
%</jawltxdoc>
%\fi

%
% Documentation:
%    (a) Without write18 enabled:
%          pdflatex uealttr.dtx
%          makeindex -s gind.ist uealttr.idx
%          makeindex -s gglo.ist -o uealttr.gls uealttr.glo
%          pdflatex uealttr.dtx
%          pdflatex uealttr.dtx
%    (b) With write18 enabled:
%          pdflatex uealttr.dtx
%          pdflatex uealttr.dtx
%          pdflatex uealttr.dtx
%
% Installation:
%     Copy uealttr.sty to a location searched by TeX, and if required
%     by your TeX installation, run the appropriate command to build
%     a hash of files (texhash, initexmf --update-fndb, etc.)
%
% Note:
%     The jawltxdoc.sty file is not needed for installation, only for
%     building the documentation; it may be deleted after producing
%     the documentation (if necessary).
%
%<*ignore>
% This is all taken verbatim from Heiko Oberdiek's packages
\begingroup
  \def\x{LaTeX2e}%
\expandafter\endgroup
\ifcase 0\ifx\install y1\fi\expandafter
         \ifx\csname processbatchFile\endcsname\relax\else1\fi
         \ifx\fmtname\x\else 1\fi\relax
\else\csname fi\endcsname
%</ignore>
%<*install>
\input docstrip.tex
\keepsilent
\askforoverwritefalse
\preamble
 ----------------------------------------------------------------
 The uealttr package --- A letter class for UEA
 Maintained by Joseph Wright
 E-mail: joseph.wright@uea.ac.uk
 Released under the LaTeX Project Public License v1.3c or later
 See http://www.latex-project.org/lppl.txt
 ----------------------------------------------------------------

\endpreamble
\Msg{Generating uealttr files:}
\generate{\file{jawltxdoc.sty}{\from{\jobname.dtx}{jawltxdoc}}
}
\usedir{tex/latex/uealttr}
\generate{\file{\jobname.cls}{\from{\jobname.dtx}{class}}
}
\usedir{source/latex/uealttr}
\generate{\file{\jobname.ins}{\from{\jobname.dtx}{install}}
}
\nopreamble\nopostamble
\usedir{doc/latex/uealttr}
\generate{\file{README.txt}{\from{\jobname.dtx}{readme}}
}
\endbatchfile
%</install>
%<*readme>
----------------------------------------------------------------
The uealttr package --- A letter class for UEA
Maintained by Joseph Wright
E-mail: joseph.wright@uea.ac.uk
Released under the LaTeX Project Public License v1.3c or later
See http://www.latex-project.org/lppl.txt
----------------------------------------------------------------

The uealttr class is version of the standard LaTeX letter class
customised for use at the University of East Anglia (UEA).  It
is based on the Word template made available by the Publications
Office.  Although aimed at UEA, the class is readily adapted to
other organisations.
%</readme>
%<*ignore>
\fi
% Will Robertson's trick
\immediate\write18{makeindex -s gind.ist -o \jobname.ind  \jobname.idx}
\immediate\write18{makeindex -s gglo.ist -o \jobname.gls  \jobname.glo}
%</ignore>
%<*driver>
\PassOptionsToClass{a4paper}{article}
\documentclass[german,english,UKenglish]{ltxdoc}
\EnableCrossrefs
\CodelineIndex
\RecordChanges
%\OnlyDescription
\usepackage{jawltxdoc}
\begin{document}
  \DocInput{\jobname.dtx}
\end{document}
%</driver>
% \fi
%
%\DoNotIndex{\&,\@date,\@empty,\\,\AddToShipoutPicture,\baselineskip}
%\DoNotIndex{\begin,\ClassInfo,\ClassWarning,\DeclareBoolOption}
%\DoNotIndex{\DeclareComplementaryOption,\def,\else,\end,\faculty}
%\DoNotIndex{\familydefault,\fi,\footnotesize,\fromaddress,\fromname}
%\DoNotIndex{\fromsig,\fromsignemail,\hspace,\IfFileExists,\ifpdf}
%\DoNotIndex{\ifx,\ignorespaces,\includegraphics,\InputIfFileExists}
%\DoNotIndex{\item,\itemindent,\leftmargin,\listparindent,\LoadClass}
%\DoNotIndex{\MessageBreak,\NeedsTeXFormat,\newcommand}
%\DoNotIndex{\newenvironment,\nobreak,\noindent,\overfullrule,\par}
%\DoNotIndex{\parindent,\parsep,\parskip,\ProcessKeyvalOptions}
%\DoNotIndex{\ProvidesClass,\put,\raggedright,\renewcomman}
%\DoNotIndex{\RequirePackage,\rightmargin,\setkeys,\setlength}
%\DoNotIndex{\SetupKeyvalOptions,\sfdefault,\stopbreaks,\strut}
%\DoNotIndex{\textbf,\thispagestyl,\toaddress,\toname,\topsep}
%\DoNotIndex{\unitlength,\vfill,\vspace,\z@,\MakeUppercase}
%\DoNotIndex{\renewcommand,\thispagestyle,\expandafter,\bfseries}
%
%\CheckSum{356}
%
% \CharacterTable
%  {Upper-case    \A\B\C\D\E\F\G\H\I\J\K\L\M\N\O\P\Q\R\S\T\U\V\W\X\Y\Z
%   Lower-case    \a\b\c\d\e\f\g\h\i\j\k\l\m\n\o\p\q\r\s\t\u\v\w\x\y\z
%   Digits        \0\1\2\3\4\5\6\7\8\9
%   Exclamation   \!     Double quote  \"     Hash (number) \#
%   Dollar        \$     Percent       \%     Ampersand     \&
%   Acute accent  \'     Left paren    \(     Right paren   \)
%   Asterisk      \*     Plus          \+     Comma         \,
%   Minus         \-     Point         \.     Solidus       \/
%   Colon         \:     Semicolon     \;     Less than     \<
%   Equals        \=     Greater than  \>     Question mark \?
%   Commercial at \@     Left bracket  \[     Backslash     \\
%   Right bracket \]     Circumflex    \^     Underscore    \_
%   Grave accent  \`     Left brace    \{     Vertical bar  \|
%   Right brace   \}     Tilde         \~}
%
%\def\GetSVNId$#1: #2.#3 #4 #5-#6-#7 #8 #9${%
%  \def\fileversion{v1.0a}%
%  \def\filedate{#5/#6/#7}}
%
%\GetSVNId $Id: uealttr.dtx 14 2008-07-23 22:03:33Z joseph $
%
%\changes{v1.0}{2008/07/21}{First public release}
%
%\setkeys{lst}{language=[LaTeX]{TeX},moretexcs={name,faculty,address,
%  department,email,phone,fax,web,position,closing,opening,logo,
%  subject,RequirePackage}}
%
%\title{\currpkg\ --- A letter class for UEA^^A
%  \thanks{This file describes version \fileversion, last revised
%    \filedate.}}
%\author{Joseph Wright^^A
%  \thanks{E-mail: joseph.wright@uea.ac.uk}}
%\date{Released \filedate}
%
%\maketitle
%
%\begin{abstract}
% The \currpkg class is version of the standard \LaTeX\ letter class
% customised for use at the University of East Anglia (UEA).  It is
% based on the \href
% {http://www1.uea.ac.uk/cm/home/services/units/mac/comm/publicationsoffice/Templates}
% {Word template} made available by the Publications Office. Although
% aimed at UEA, the class is readily adapted to other organisations.
%\end{abstract}
%
%\begin{multicols}{2}
%  \tableofcontents
%\end{multicols}
%
%\section{Introduction}
% The \currpkg class is based on the standard \LaTeX\ class
% \pkg{letter}.  It therefore inherits all of the normal macros from
% the parent: \cs{name}, \cs{opening}, \cs{closing}, \etc.  However,
% the class follows the current guidelines given by UEA for official
% letters. This makes use of a number of additional data macros, and
% also allows ready customisation.  It also makes layout changes to
% include a logo and address information.
%
%\section{Using the class}
%
%\DescribeOption{draft}
%\DescribeOption{final}
% The class is loaded in the usual way, as the argument to
% \cs{documentclass}. The package recognises the \opt{draft} option,
% which will result in the inclusion of thick black bars to show
% overfull boxes.  Any graphics will still be included, as
% \pkg{graphicx} is loaded with the \opt{final} option.
%
%\DescribeMacro{\logo}
% To allow setting up of a graphical logo, the \cs{logo} macro is
% provided by the package.  This is used to set the name  of the file
% containing the logo.  To allow use both with \LaTeX\ and
% pdf\LaTeX\, this macro should not include the file extension. Like
% other \pkg{letter} macros, \cs{logo} takes a single argument.
%\begin{LaTeXexample}[noexample]
%  \logo{uealogo}
%\end{LaTeXexample}
% will therefore cause the class to look for \file{uealogo.eps} if
% compilation uses \LaTeX, or \file{uealogo.pdf} if using pdf\LaTeX.
% The default setting of \cs{logo} is \opt{uealogo}.
%
% For UEA users, the official logo is available as a \ext{eps} file
% from the \href
% {http://www1.uea.ac.uk/cm/home/services/units/mac/comm/publicationsoffice/Logos}
% {Publications Office}. The file can be converted to a \ext{pdf}
% using \pkg{epstopdf}.  Doing this and saving both files to the
% \TeX\ path will allow compilation with either \LaTeX\ or pdf\LaTeX.
%
%\DescribeOption{logo}
% The class is designed so that the first page printed always
% contains space for the logo.  Second and subsequent pages are
% adjusted so that more of the paper is used for printing and the
% logo is not required. The option \opt{logo} governs whether
% the class attempts to print the logo, or simply reserves the space.
% The option takes the values \opt{true} and \opt{false}, using the
% key--value method.  To prevent printing the logo, the class is
% loaded as follows.
%\begin{LaTeXexample}[noexample]
%  \documentclass[logo=false]{uealttr}
%\end{LaTeXexample}
% Note that by default, the class prints the logo (\ie as if
% \opt{logo=true} had been given).
%
%\DescribeOption{personal}
%\DescribeOption{confidential}
% The \opt{personal} and \opt{confidential} options are provided.
% These take Boolean (true/false) values using key--value syntax,
% but can also be given alone.  Thus
%\begin{LaTeXexample}[noexample]
%  \documentclass[personal]{uealttr}
%\end{LaTeXexample}
% and
%\begin{LaTeXexample}[noexample]
%  \documentclass[personal=true]{uealttr}
%\end{LaTeXexample}
% act in the same way.  The two options will include ``PERSONAL'',
% ``CONFIDENTIAL'' or ``PERSONAL \& CONFIDENTIAL'' in the address
% area, if required.
%
%\DescribeMacro{\subject}
%\changes{v1.0a}{2008/07/23}{Altered \cs{subject} macro to alter
%  style used by the Registry}
%\DescribeMacro{\faculty}
%\DescribeMacro{\department}
% A number of pieces of data can be gathered by the standard
% \pkg{letter} class, in macros such as \cs{name}, \cs{address},
% \etc. The \currpkg package adds a number of macros to this list,
% all of which should be given before \cs{opening}. The \cs{subject}
% macro is used to place a subject line in the output. The
% \cs{faculty} and \cs{department} macros include the obvious
% information into the output file, before the contents of
% \cs{address}.
%
%\DescribeMacro{\email}
%\DescribeMacro{\phone}
%\DescribeMacro{\fax}
%\DescribeMacro{\web}
% The macros \cs{email}, \cs{phone} and \cs{fax} are used to include
% general contact details underneath the address area.  In the same
% way, \cs{web} includes a website in the same part of the letter.
% This information will often be general departmental contact
% details.
%\DescribeMacro{\position}
%\DescribeMacro{\signemail}
%\DescribeMacro{\signphone}
% In contrast, \cs{position}, \cs{signemail} and \cs{signphone} add
% information under the signature.  Thus these are intended to relate
% to the person signing the letter.  Notice that the name for the
% signature is taken from \cs{signature} if available, otherwise the
% \cs{name} macro is used.
%
%\section{A demonstration letter}
%
% A simple letter, with all of the data directly in the source, might
% read as follows.
%\begin{LaTeXexample}[noexample]
%  \documentclass[english,UKenglish]{uealttr}
%  \usepackage[final]{microtype}
%  \usepackage{babel}
%  \name{Joseph Wright}
%  \faculty{Faculty of Science}
%  \department{School of Chemical Sciences and Pharmacy}
%  \address{
%    University of East Anglia \\
%    Norwich NR4 7TJ \\
%    United Kingdom}
%  \email{joseph.wright@uea.ac.uk}
%  \phone{+44 (0)1603 591680}
%  \fax{+44 (0)1603 592044}
%  \web{www.uea.ac.uk}
%  \position{Senior Research Associate}
%  \begin{document}
%  \begin{letter}
%  {Mr.~A.N.~Other \\ Some Company \\ Some Street \\ Sometown}
%  \subject{A demonstration letter}
%  \opening{Dear Mr.~Other,}
%
%  This is a rather boring letter, which simply shows how to use
%  the class file.
%
%  \closing{Yours faithfully,}
%  \end{letter}
%  \end{document}
%\end{LaTeXexample}
%
% To make configuration easier, the class will attempt to load a
% configuration file \file{uealttr.cfg}.  This can be used to set up
% repeated data. This can also contain other instructions for \LaTeX.
%  For example, to include the standard data above in every letter,
% the class author uses a configuration file reading
%\begin{LaTeXexample}[noexample]
%  \name{Joseph Wright}
%  \faculty{Faculty of Science}
%  \department{School of Chemical Sciences and Pharmacy}
%  \address{
%    University of East Anglia \\
%    Norwich NR4 7TJ \\
%    United Kingdom}
%  \email{joseph.wright@uea.ac.uk}
%  \phone{+44 (0)1603 591680}
%  \fax{+44 (0)1603 592044}
%  \web{www.uea.ac.uk}
%  \position{Senior Research Associate}
%  \RequirePackage{microtype}
%  \RequirePackage{babel}
%\end{LaTeXexample}
%
% The letter can then be reduced to.
%\begin{LaTeXexample}[noexample]
%  \documentclass[english,UKenglish]{uealttr}
%  \begin{document}
%  \begin{letter}
%  {Mr.~A.N.~Other \\ Some Company \\ Some Street \\ Sometown}
%  \subject{A demonstration letter}
%  \opening{Dear Mr.~Other,}
%
%  This is a rather boring letter, which simply shows how to use
%  the class file.
%
%  \closing{Yours faithfully,}
%  \end{letter}
%  \end{document}
%\end{LaTeXexample}
%
%\StopEventually{%
%  \PrintChanges
%  \PrintIndex}
%
%\iffalse
%<*class>
%\fi
%
%\section{The code}
%\begin{macro}{\uea@id}
% The package starts with the usual identification.
%    \begin{macrocode}
\NeedsTeXFormat{LaTeX2e}
\def\uea@id$#1: #2.#3 #4 #5-#6-#7 #8 #9${#5/#6/#7}
\ProvidesClass{uealttr}
  [\uea@id$Id: uealttr.dtx 14 2008-07-23 22:03:33Z joseph $
   v1.0a A letter class for UEA]
%    \end{macrocode}
%\end{macro}
% The standard support packages are loaded.
%    \begin{macrocode}
\LoadClass[10pt,a4paper]{letter}
\RequirePackage[T1]{fontenc}
\RequirePackage[final]{graphicx}
\RequirePackage[parfill]{parskip}
\RequirePackage{helvet,eso-pic,ifpdf,kvoptions}
\RequirePackage[
  hmargin=30mm,
  vmargin=25mm,
  dvips]{geometry}
%    \end{macrocode}
%\begin{macro}{\ifuea@personal}
%\begin{macro}{\ifuea@confidential}
%\begin{macro}{\ifuea@logo}
%\begin{macro}{\ifuea@draft}
% The single package option is declared.
%    \begin{macrocode}
\SetupKeyvalOptions{
  family = uea,
  prefix = uea@}
\DeclareBoolOption{personal}
\DeclareBoolOption{confidential}
\DeclareBoolOption{logo}
\DeclareBoolOption{draft}
\DeclareComplementaryOption{final}{draft}
\setkeys{uea}{
  personal = false,
  confidential = false,
  logo = true,
  draft = false}
\ProcessKeyvalOptions{uea}
\ifuea@draft
  \setlength\overfullrule{5pt}
\else
  \setlength\overfullrule{0pt}
\fi
%    \end{macrocode}
%\end{macro}
%\end{macro}
%\end{macro}
%\end{macro}
%\begin{macro}{\uea@pandc}
% For personal and confidential letters, some text is set up here.
%    \begin{macrocode}
\newcommand*{\uea@pandc}{}
\ifuea@personal
  \renewcommand*{\uea@pandc}{PERSONAL}
  \ifuea@confidential
    \renewcommand*{\uea@pandc}{PERSONAL \& CONFIDENTIAL}
  \fi
\else
  \ifuea@confidential
    \renewcommand*{\uea@pandc}{CONFIDENTIAL}
  \fi
\fi
%    \end{macrocode}
%\end{macro}
% The UEA style is to use Helvetica for all text.
%    \begin{macrocode}
\renewcommand{\familydefault}{\sfdefault}
%    \end{macrocode}
%\begin{macro}{\subject}
%\begin{macro}{\lettersubject}
% The subject of the letter is set up.
%    \begin{macrocode}
\newcommand*{\subject}[1]{\def\lettersubject{#1}}
\subject{}
%    \end{macrocode}
%\end{macro}
%\end{macro}
%\begin{macro}{\faculty}
%\begin{macro}{\fromfaculty}
%\begin{macro}{\department}
%\begin{macro}{\fromdept}
% A number of macros are used to store the various pieces of data
% used in the address block.  First the faculty and department.
%    \begin{macrocode}
\newcommand*{\faculty}[1]{\def\fromfaculty{#1}}
\newcommand*{\department}[1]{\def\fromdept{#1}}
\faculty{}
\department{}
%    \end{macrocode}
%\end{macro}
%\end{macro}
%\end{macro}
%\end{macro}
%\begin{macro}{\email}
%\begin{macro}{\fromemail}
%\begin{macro}{\phone}
%\begin{macro}{\fromphone}
%\begin{macro}{\fax}
%\begin{macro}{\fromfax}
%\begin{macro}{\web}
%\begin{macro}{\fromweb}
% Next come the contact details for the right-hand area.
%    \begin{macrocode}
\newcommand*{\email}[1]{\def\fromemail{#1}}
\newcommand*{\phone}[1]{\def\fromphone{#1}}
\newcommand*{\fax}[1]{\def\fromfax{#1}}
\newcommand*{\web}[1]{\def\fromweb{#1}}
\email{}
\phone{}
\fax{}
\web{}
%    \end{macrocode}
%\end{macro}
%\end{macro}
%\end{macro}
%\end{macro}
%\end{macro}
%\end{macro}
%\end{macro}
%\end{macro}
%\begin{macro}{\position}
%\begin{macro}{\fromposition}
%\begin{macro}{\signphone}
%\begin{macro}{\fromsignphone}
%\begin{macro}{\signemail}
%\begin{macro}{\fromsignemail}
% Finally, some details added after the signature.
%    \begin{macrocode}
\newcommand*{\position}[1]{\def\fromposition{#1}}
\newcommand*{\signphone}[1]{\def\fromsignphone{#1}}
\newcommand*{\signemail}[1]{\def\fromsignemail{#1}}
\position{}
\signphone{}
\signemail{}
%    \end{macrocode}
%\end{macro}
%\end{macro}
%\end{macro}
%\end{macro}
%\end{macro}
%\end{macro}
%\begin{macro}{\logo}
%\begin{macro}{\fromlogo}
% Any local configuration is loaded, and options are processed.
%    \begin{macrocode}
\newcommand*{\logo}[1]{\def\fromlogo{#1}}
\logo{uealogo}
\InputIfFileExists{uealttr.cfg}
  {\ClassInfo{uealttr}{Loaded local configuration file}}
  {}
%    \end{macrocode}
%\end{macro}
%\end{macro}
% If no logo is available, the package will skip trying to position
% it.
%    \begin{macrocode}
\ifpdf
  \IfFileExists{\fromlogo.pdf}
    {}
    {\ifuea@logo
       \ClassWarning{uealttr}
         {Logo file \fromlogo.pdf not found!\MessageBreak
          No logo will be included in output}
     \fi
     \uea@logofalse}
\else
  \IfFileExists{\fromlogo.eps}
    {}
    {\ifuea@logo
       \ClassWarning{uealttr}
         {Logo file \fromlogo.eps not found!\MessageBreak
          No logo will be included in output}
     \fi
     \uea@logofalse}
\fi
%    \end{macrocode}
%\begin{environment}{wider}
% A trick taken from \pkg{memoir} and the UK FAQ: here, the extra
% width needed is known.
%    \begin{macrocode}
\newenvironment{wider}{%
  \begin{list}{}{%
    \setlength{\topsep}{0pt}%
    \setlength{\leftmargin}{\z@}%
    \setlength{\rightmargin}{-22mm}%
    \setlength{\listparindent}{\parindent}%
    \setlength{\itemindent}{\parindent}%
    \setlength{\parsep}{\parskip}%
  }%
  \item[]}{\end{list}}

%    \end{macrocode}
%\end{environment}
%\begin{macro}{\opening}
% To achieve the correct layout, completely new \cs{opening}
% and \cs{closing} macros are employed.
%    \begin{macrocode}
\renewcommand*{\opening}[1]{%
  \thispagestyle{empty}%
  \vspace*{19mm}%
%    \end{macrocode}
% If the logo has been requested, it is included here.
%    \begin{macrocode}
  \ifuea@logo
    \AddToShipoutPicture*{%
      \setlength{\unitlength}{1mm}%
      \put(147,257){\includegraphics{\fromlogo}}}%
  \fi
%    \end{macrocode}
% The address block is created by using a series of minipages of the
% correct width.  All of this is nested inside a group where the
% margins are made wider.
%    \begin{macrocode}
  \begin{wider}
    \setlength{\leftmargin}{\z@}%
    \setlength{\rightmargin}{22mm}%
    \begin{minipage}[t]{80mm}
      \begin{minipage}[t]{80mm}
        \raggedright
        \ifx\@empty\uea@pandc\else
          \textbf{\uea@pandc} \\*[\baselineskip]
        \fi
        \toname \\
        \toaddress
      \end{minipage}%
%    \end{macrocode}
% A zero-width minipage is used to force the date downward to the
% desired position.
%    \begin{macrocode}
      \begin{minipage}[t]{0mm}
        \vspace*{35mm}%
      \end{minipage}
      \@date
    \end{minipage}%
%    \end{macrocode}
% This minipage generates the whitespace between the two address
% blocks.
%    \begin{macrocode}
    \begin{minipage}[t]{42mm}
      \hspace{42mm}%
      \vspace*{55mm}%
    \end{minipage}%
%    \end{macrocode}
% For the address block, a series of checks are made so that no empty
% lines are ended with |\\|.
%    \begin{macrocode}
    \begin{minipage}[t]{50mm}
      \footnotesize
      \raggedright
      \ifx\@empty\fromfaculty
        \ifx\@empty\fromdept\else
          \ignorespaces\fromdept \\*[\baselineskip]
        \fi
      \else
        \ifx\@empty\fromdept
          \textbf{\ignorespaces\fromfaculty} \\*[\baselineskip]
        \else
          \textbf{\ignorespaces\fromfaculty} \\
          \ignorespaces\fromdept \\*[\baselineskip]
        \fi
      \fi
      \ifx\@empty\fromaddress\else
        \ignorespaces\fromaddress \\*[\baselineskip]
      \fi
      \ifx\@empty\fromemail\else
        Email: \ignorespaces\fromemail \\
      \fi
      \ifx\@empty\fromphone\else
        Tel: \ignorespaces\fromphone \\
      \fi
      \ifx\@empty\fromfax\else
        Fax: \ignorespaces\fromfax \\
      \fi
      \ifx\@empty\fromweb\else
        \ignorespaces\fromweb \\
      \fi
    \end{minipage}%
  \end{wider}
  \par\noindent#1\par\nobreak
  {\bfseries
   \ifx\@empty\lettersubject\else
     \expandafter\uea@MakeUppercase\lettersubject \\\@empty
   \fi}}
%    \end{macrocode}
%\end{macro}
%\begin{macro}{\uea@MakeUppercase}
% A support macro is needed to allow multi-line uppercase.
%    \begin{macrocode}
\def\uea@MakeUppercase#1\\#2\@empty{%
  \MakeUppercase{#1}%
  \ifx\@empty#2\@empty\else
    \\ \expandafter\uea@MakeUppercase#2\@empty
  \fi}
%    \end{macrocode}
%\end{macro}
%\begin{macro}{\closing}
%\darg{closing}
% The same tricks are used for the closing.
%    \begin{macrocode}
\renewcommand{\closing}[1]{%
  \par\nobreak\vspace{\parskip}%
  \stopbreaks
  \noindent
  \ignorespaces #1\\[24mm]
  \ifx\@empty\fromsig
    \fromname \\
  \else
    \fromsig \\
  \fi
  \ifx\@empty\fromposition\else
    \fromposition \\
  \fi
  \ifx\@empty\fromsignphone\else
    \fromsignphone \\
  \fi
  \ifx\@empty\fromsignemail\else
    \fromsignemail \\
  \fi
  \strut
  \par}
%    \end{macrocode}
%\end{macro}
%\begin{macro}{\stopletter}
% The \cs{stopletter} macro is used to push the contents up the page,
% in contrast to normal \LaTeX\ behaviour.
%    \begin{macrocode}
\renewcommand*{\stopletter}{\vfill}
%    \end{macrocode}
%\end{macro}
%
%\Finale
%\iffalse
%</class>
%<*jawltxdoc>
\NeedsTeXFormat{LaTeX2e}
\ProvidesPackage{jawltxdoc}
\usepackage[T1]{fontenc}
\usepackage{lmodern}
\usepackage[final]{listings,graphicx,microtype}
\usepackage[scaled=0.95]{helvet}
\usepackage[version=3]{mhchem}
\usepackage[osf]{mathpazo}
\usepackage{booktabs,array,url,courier,xspace,varioref}
\usepackage{upgreek,ifpdf,float,caption,longtable,babel}
\begingroup
  \@ifundefined{eTeXversion}
    {\aftergroup\@gobble}
    {\aftergroup\@firstofone}
\endgroup
{\usepackage{etoolbox}}
\floatstyle{plaintop}
\restylefloat{table}
\labelformat{figure}{\figurename~#1}
\labelformat{table}{\tablename~#1}
\ifpdf
  \usepackage{embedfile}
  \embedfile[%
    stringmethod=escape,%
    mimetype=plain/text,%
    desc={LaTeX docstrip source archive for package `\jobname'}%
    ]{\jobname.dtx}
\fi
\IfFileExists{\jobname.sty}
  {\usepackage{\jobname}}{}
\usepackage[numbered]{hypdoc}
\setcounter{IndexColumns}{2}
\newlength\LaTeXwidth
\newlength\LaTeXoutdent
\newlength\LaTeXgap
\setlength\LaTeXgap{1em}
\setlength\LaTeXoutdent{-0.15\textwidth}
\newbox\lst@samplebox
\edef\LaTeXexamplefile{\jobname.tmp}
\lst@RequireAspects{writefile}
\lstnewenvironment{LaTeXexample}[1][example]{%
  \global\let\lst@intname\@empty
  \ifcsname LaTeXcode#1\endcsname
    \expandafter\let\expandafter\LaTeXcode
      \csname LaTeXcode#1\endcsname
    \expandafter\let\expandafter\LaTeXcodeend
      \csname LaTeXcode#1end\endcsname
  \else
    \PackageError{jawltxdoc}
      {Undefined example type `#1'}
      \@ehd
    \let\LaTeXcode\relax
    \let\LaTeXcodeend\relax
  \fi
  \LaTeXcode}
  {\lst@EndWriteFile
   \LaTeXcodeend}
\newcommand*{\LaTeXcodeexample}{%
  \setbox\lst@samplebox=\hbox\bgroup
  \LaTeXcodefloat}
\let\LaTeXcoderesultonly\LaTeXcodeexample
\newcommand*{\LaTeXcodeexampleend}{%
  \egroup
  \setlength\LaTeXwidth{\wd\lst@samplebox}%
  \begin{list}{}{%
    \setlength\itemindent{0pt}
    \setlength\leftmargin\LaTeXoutdent
    \setlength\rightmargin{0pt}}%
    \item
      \setlength\LaTeXoutdent{-0.15\textwidth}
      \begin{minipage}[c]{%
        \textwidth-\LaTeXwidth-\LaTeXoutdent-\LaTeXgap}
        \LaTeXcodefloatend
      \end{minipage}%
      \hfill
      \begin{minipage}[c]{\LaTeXwidth}%
        \hbox to\linewidth{\box\lst@samplebox\hss}%
      \end{minipage}%
  \end{list}}
\newcommand*{\LaTeXcodefloat}{%
  \setkeys{lst}{tabsize=4,gobble=3,breakindent=0pt,
    basicstyle=\small\ttfamily,basewidth=0.51em,
    keywordstyle=\color{blue}}%
  \lst@BeginAlsoWriteFile{\LaTeXexamplefile}}
\let\LaTeXcodenoexample\LaTeXcodefloat
\let\LaTeXcodenoexampleend\@empty
\newcommand*{\LaTeXcodefloatend}{%
  \MakePercentComment\catcode`\^^M=10\relax
  \small
  {\setkeys{lst}{SelectCharTable=\lst@ReplaceInput{\^\^I}%
    {\lst@ProcessTabulator}}%
    \leavevmode \input{\LaTeXexamplefile}}%
  \MakePercentIgnore}
\newcommand*{\LaTeXcoderesultonlyend}{\egroup\LaTeXcodefloatend}
\lstnewenvironment{BibTeXexample}{%
  \global\let\lst@intname\@empty
  \setbox\lst@samplebox=\hbox\bgroup
  \setkeys{lst}{tabsize=4,gobble=3,breakindent=0pt,
    basicstyle=\small\ttfamily,basewidth=0.51em,
    keywordstyle=\color{black}}
  \lst@BeginAlsoWriteFile{\LaTeXexamplefile}}
 {\lst@EndWriteFile
   \LaTeXcodeexampleend}
\newcommand*{\DescribeOption}{%
  \leavevmode\@bsphack\begingroup\MakePrivateLetters
  \Describe@Option}
\newcommand*{\Describe@Option}[1]{\endgroup
              \marginpar{\raggedleft\PrintDescribeEnv{#1}}%
              \SpecialOptionIndex{#1}\@esphack\ignorespaces}
\newcommand*{\SpecialOptionIndex}[1]{\@bsphack
    \index{#1\actualchar{\protect\ttfamily#1}
           (option)\encapchar usage}%
    \index{options:\levelchar#1\actualchar{\protect\ttfamily#1}%
      \encapchar usage}\@esphack}
\newcommand*{\indexopt}[1]{\DescribeOption{#1}\opt{#1}}
\newcommand*{\DescribeOptionInfo}[2]{%
  \DescribeOption{#1}%
  \opt{#1=\meta{#2}}\xspace}
\newcommand*{\ofixarg}[1]{%
  {\ttfamily[}%
  \ifmmode \expandafter \nfss@text \fi
  {%
    \meta@font@select
    \edef\meta@hyphen@restore{%
      \hyphenchar\the\font\the\hyphenchar\font}%
    \hyphenchar\font\m@ne
    \language\l@nohyphenation
    #1\/%
    \meta@hyphen@restore
    }%
    {\ttfamily]}}
\newcommand*{\pkg}[1]{\textsf{#1}}
\newcommand*{\currpkg}{\pkg{\jobname}\xspace}
\newcommand*{\opt}[1]{\texttt{#1}}
\newcommand*{\defaultopt}[1]{\opt{\textbf{#1}}}
\newcommand*{\file}[1]{\texttt{#1}}
\newcommand*{\ext}[1]{\file{.#1}}
\newcommand*{\latin}[1]{\emph{#1}}
\newcommand*{\etc}{%
  \@ifnextchar.
    {\latin{etc}}
    {\latin{etc}.\xspace}}
\newcommand*{\eg}{%
  \@ifnextchar.
    {\latin{e.g}}
    {\latin{e.g}.\xspace}}
\newcommand*{\ie}{%
  \@ifnextchar.
    {\latin{i.e}}
    {\latin{i.e}.\xspace}}
\newcommand*{\etal}{%
  \@ifnextchar.
    {\latin{et~al.}}
    {\latin{et~al}.\xspace}}
\newcommand*{\AMS}{{\protect\usefont{OMS}{cmsy}{m}{n}%
  A\kern-.1667em\lower.5ex\hbox{M}\kern-.125emS}}
\providecommand*{\eTeX}{\ensuremath{\varepsilon}-\TeX}
\DeclareRobustCommand*{\XeTeX}
  {X\kern-.125em\lower.5ex\hbox{\reflectbox{E}}\kern-.1667em\TeX}
\providecommand*{\CTAN}{\textsc{ctan}}
\@ifpackageloaded{etoolbox}
  {\patchcmd{\@addmarginpar}
    {\@latex@warning@no@line {Marginpar on page \thepage\space moved}}
    {\relax}{}{}}
  {}
\newcounter{argument}
\g@addto@macro\endmacro{\setcounter{argument}{0}}
\newcommand*\darg[1]{%
  \stepcounter{argument}%
  {\ttfamily\char`\#\theargument~:~}#1\par\noindent\ignorespaces}
\newcommand*\doarg[1]{%
  \stepcounter{argument}%
  {\ttfamily\makebox[0pt][r]{[}%
   \char`\#\theargument]:~}#1\par\noindent\ignorespaces}
%</jawltxdoc>
%\fi

%
% Documentation:
%    (a) Without write18 enabled:
%          pdflatex uealttr.dtx
%          makeindex -s gind.ist uealttr.idx
%          makeindex -s gglo.ist -o uealttr.gls uealttr.glo
%          pdflatex uealttr.dtx
%          pdflatex uealttr.dtx
%    (b) With write18 enabled:
%          pdflatex uealttr.dtx
%          pdflatex uealttr.dtx
%          pdflatex uealttr.dtx
%
% Installation:
%     Copy uealttr.sty to a location searched by TeX, and if required
%     by your TeX installation, run the appropriate command to build
%     a hash of files (texhash, initexmf --update-fndb, etc.)
%
% Note:
%     The jawltxdoc.sty file is not needed for installation, only for
%     building the documentation; it may be deleted after producing
%     the documentation (if necessary).
%
%<*ignore>
% This is all taken verbatim from Heiko Oberdiek's packages
\begingroup
  \def\x{LaTeX2e}%
\expandafter\endgroup
\ifcase 0\ifx\install y1\fi\expandafter
         \ifx\csname processbatchFile\endcsname\relax\else1\fi
         \ifx\fmtname\x\else 1\fi\relax
\else\csname fi\endcsname
%</ignore>
%<*install>
\input docstrip.tex
\keepsilent
\askforoverwritefalse
\preamble
 ----------------------------------------------------------------
 The uealttr package --- A letter class for UEA
 Maintained by Joseph Wright
 E-mail: joseph.wright@uea.ac.uk
 Released under the LaTeX Project Public License v1.3c or later
 See http://www.latex-project.org/lppl.txt
 ----------------------------------------------------------------

\endpreamble
\Msg{Generating uealttr files:}
\generate{\file{jawltxdoc.sty}{\from{\jobname.dtx}{jawltxdoc}}
}
\usedir{tex/latex/uealttr}
\generate{\file{\jobname.cls}{\from{\jobname.dtx}{class}}
}
\usedir{source/latex/uealttr}
\generate{\file{\jobname.ins}{\from{\jobname.dtx}{install}}
}
\nopreamble\nopostamble
\usedir{doc/latex/uealttr}
\generate{\file{README.txt}{\from{\jobname.dtx}{readme}}
}
\endbatchfile
%</install>
%<*readme>
----------------------------------------------------------------
The uealttr package --- A letter class for UEA
Maintained by Joseph Wright
E-mail: joseph.wright@uea.ac.uk
Released under the LaTeX Project Public License v1.3c or later
See http://www.latex-project.org/lppl.txt
----------------------------------------------------------------

The uealttr class is version of the standard LaTeX letter class
customised for use at the University of East Anglia (UEA).  It
is based on the Word template made available by the Publications
Office.  Although aimed at UEA, the class is readily adapted to
other organisations.
%</readme>
%<*ignore>
\fi
% Will Robertson's trick
\immediate\write18{makeindex -s gind.ist -o \jobname.ind  \jobname.idx}
\immediate\write18{makeindex -s gglo.ist -o \jobname.gls  \jobname.glo}
%</ignore>
%<*driver>
\PassOptionsToClass{a4paper}{article}
\documentclass[german,english,UKenglish]{ltxdoc}
\EnableCrossrefs
\CodelineIndex
\RecordChanges
%\OnlyDescription
\usepackage{jawltxdoc}
\begin{document}
  \DocInput{\jobname.dtx}
\end{document}
%</driver>
% \fi
%
%\DoNotIndex{\&,\@date,\@empty,\\,\AddToShipoutPicture,\baselineskip}
%\DoNotIndex{\begin,\ClassInfo,\ClassWarning,\DeclareBoolOption}
%\DoNotIndex{\DeclareComplementaryOption,\def,\else,\end,\faculty}
%\DoNotIndex{\familydefault,\fi,\footnotesize,\fromaddress,\fromname}
%\DoNotIndex{\fromsig,\fromsignemail,\hspace,\IfFileExists,\ifpdf}
%\DoNotIndex{\ifx,\ignorespaces,\includegraphics,\InputIfFileExists}
%\DoNotIndex{\item,\itemindent,\leftmargin,\listparindent,\LoadClass}
%\DoNotIndex{\MessageBreak,\NeedsTeXFormat,\newcommand}
%\DoNotIndex{\newenvironment,\nobreak,\noindent,\overfullrule,\par}
%\DoNotIndex{\parindent,\parsep,\parskip,\ProcessKeyvalOptions}
%\DoNotIndex{\ProvidesClass,\put,\raggedright,\renewcomman}
%\DoNotIndex{\RequirePackage,\rightmargin,\setkeys,\setlength}
%\DoNotIndex{\SetupKeyvalOptions,\sfdefault,\stopbreaks,\strut}
%\DoNotIndex{\textbf,\thispagestyl,\toaddress,\toname,\topsep}
%\DoNotIndex{\unitlength,\vfill,\vspace,\z@,\MakeUppercase}
%\DoNotIndex{\renewcommand,\thispagestyle,\expandafter,\bfseries}
%
%\CheckSum{356}
%
% \CharacterTable
%  {Upper-case    \A\B\C\D\E\F\G\H\I\J\K\L\M\N\O\P\Q\R\S\T\U\V\W\X\Y\Z
%   Lower-case    \a\b\c\d\e\f\g\h\i\j\k\l\m\n\o\p\q\r\s\t\u\v\w\x\y\z
%   Digits        \0\1\2\3\4\5\6\7\8\9
%   Exclamation   \!     Double quote  \"     Hash (number) \#
%   Dollar        \$     Percent       \%     Ampersand     \&
%   Acute accent  \'     Left paren    \(     Right paren   \)
%   Asterisk      \*     Plus          \+     Comma         \,
%   Minus         \-     Point         \.     Solidus       \/
%   Colon         \:     Semicolon     \;     Less than     \<
%   Equals        \=     Greater than  \>     Question mark \?
%   Commercial at \@     Left bracket  \[     Backslash     \\
%   Right bracket \]     Circumflex    \^     Underscore    \_
%   Grave accent  \`     Left brace    \{     Vertical bar  \|
%   Right brace   \}     Tilde         \~}
%
%\def\GetSVNId$#1: #2.#3 #4 #5-#6-#7 #8 #9${%
%  \def\fileversion{v1.0a}%
%  \def\filedate{#5/#6/#7}}
%
%\GetSVNId $Id: uealttr.dtx 14 2008-07-23 22:03:33Z joseph $
%
%\changes{v1.0}{2008/07/21}{First public release}
%
%\setkeys{lst}{language=[LaTeX]{TeX},moretexcs={name,faculty,address,
%  department,email,phone,fax,web,position,closing,opening,logo,
%  subject,RequirePackage}}
%
%\title{\currpkg\ --- A letter class for UEA^^A
%  \thanks{This file describes version \fileversion, last revised
%    \filedate.}}
%\author{Joseph Wright^^A
%  \thanks{E-mail: joseph.wright@uea.ac.uk}}
%\date{Released \filedate}
%
%\maketitle
%
%\begin{abstract}
% The \currpkg class is version of the standard \LaTeX\ letter class
% customised for use at the University of East Anglia (UEA).  It is
% based on the \href
% {http://www1.uea.ac.uk/cm/home/services/units/mac/comm/publicationsoffice/Templates}
% {Word template} made available by the Publications Office. Although
% aimed at UEA, the class is readily adapted to other organisations.
%\end{abstract}
%
%\begin{multicols}{2}
%  \tableofcontents
%\end{multicols}
%
%\section{Introduction}
% The \currpkg class is based on the standard \LaTeX\ class
% \pkg{letter}.  It therefore inherits all of the normal macros from
% the parent: \cs{name}, \cs{opening}, \cs{closing}, \etc.  However,
% the class follows the current guidelines given by UEA for official
% letters. This makes use of a number of additional data macros, and
% also allows ready customisation.  It also makes layout changes to
% include a logo and address information.
%
%\section{Using the class}
%
%\DescribeOption{draft}
%\DescribeOption{final}
% The class is loaded in the usual way, as the argument to
% \cs{documentclass}. The package recognises the \opt{draft} option,
% which will result in the inclusion of thick black bars to show
% overfull boxes.  Any graphics will still be included, as
% \pkg{graphicx} is loaded with the \opt{final} option.
%
%\DescribeMacro{\logo}
% To allow setting up of a graphical logo, the \cs{logo} macro is
% provided by the package.  This is used to set the name  of the file
% containing the logo.  To allow use both with \LaTeX\ and
% pdf\LaTeX\, this macro should not include the file extension. Like
% other \pkg{letter} macros, \cs{logo} takes a single argument.
%\begin{LaTeXexample}[noexample]
%  \logo{uealogo}
%\end{LaTeXexample}
% will therefore cause the class to look for \file{uealogo.eps} if
% compilation uses \LaTeX, or \file{uealogo.pdf} if using pdf\LaTeX.
% The default setting of \cs{logo} is \opt{uealogo}.
%
% For UEA users, the official logo is available as a \ext{eps} file
% from the \href
% {http://www1.uea.ac.uk/cm/home/services/units/mac/comm/publicationsoffice/Logos}
% {Publications Office}. The file can be converted to a \ext{pdf}
% using \pkg{epstopdf}.  Doing this and saving both files to the
% \TeX\ path will allow compilation with either \LaTeX\ or pdf\LaTeX.
%
%\DescribeOption{logo}
% The class is designed so that the first page printed always
% contains space for the logo.  Second and subsequent pages are
% adjusted so that more of the paper is used for printing and the
% logo is not required. The option \opt{logo} governs whether
% the class attempts to print the logo, or simply reserves the space.
% The option takes the values \opt{true} and \opt{false}, using the
% key--value method.  To prevent printing the logo, the class is
% loaded as follows.
%\begin{LaTeXexample}[noexample]
%  \documentclass[logo=false]{uealttr}
%\end{LaTeXexample}
% Note that by default, the class prints the logo (\ie as if
% \opt{logo=true} had been given).
%
%\DescribeOption{personal}
%\DescribeOption{confidential}
% The \opt{personal} and \opt{confidential} options are provided.
% These take Boolean (true/false) values using key--value syntax,
% but can also be given alone.  Thus
%\begin{LaTeXexample}[noexample]
%  \documentclass[personal]{uealttr}
%\end{LaTeXexample}
% and
%\begin{LaTeXexample}[noexample]
%  \documentclass[personal=true]{uealttr}
%\end{LaTeXexample}
% act in the same way.  The two options will include ``PERSONAL'',
% ``CONFIDENTIAL'' or ``PERSONAL \& CONFIDENTIAL'' in the address
% area, if required.
%
%\DescribeMacro{\subject}
%\changes{v1.0a}{2008/07/23}{Altered \cs{subject} macro to alter
%  style used by the Registry}
%\DescribeMacro{\faculty}
%\DescribeMacro{\department}
% A number of pieces of data can be gathered by the standard
% \pkg{letter} class, in macros such as \cs{name}, \cs{address},
% \etc. The \currpkg package adds a number of macros to this list,
% all of which should be given before \cs{opening}. The \cs{subject}
% macro is used to place a subject line in the output. The
% \cs{faculty} and \cs{department} macros include the obvious
% information into the output file, before the contents of
% \cs{address}.
%
%\DescribeMacro{\email}
%\DescribeMacro{\phone}
%\DescribeMacro{\fax}
%\DescribeMacro{\web}
% The macros \cs{email}, \cs{phone} and \cs{fax} are used to include
% general contact details underneath the address area.  In the same
% way, \cs{web} includes a website in the same part of the letter.
% This information will often be general departmental contact
% details.
%\DescribeMacro{\position}
%\DescribeMacro{\signemail}
%\DescribeMacro{\signphone}
% In contrast, \cs{position}, \cs{signemail} and \cs{signphone} add
% information under the signature.  Thus these are intended to relate
% to the person signing the letter.  Notice that the name for the
% signature is taken from \cs{signature} if available, otherwise the
% \cs{name} macro is used.
%
%\section{A demonstration letter}
%
% A simple letter, with all of the data directly in the source, might
% read as follows.
%\begin{LaTeXexample}[noexample]
%  \documentclass[english,UKenglish]{uealttr}
%  \usepackage[final]{microtype}
%  \usepackage{babel}
%  \name{Joseph Wright}
%  \faculty{Faculty of Science}
%  \department{School of Chemical Sciences and Pharmacy}
%  \address{
%    University of East Anglia \\
%    Norwich NR4 7TJ \\
%    United Kingdom}
%  \email{joseph.wright@uea.ac.uk}
%  \phone{+44 (0)1603 591680}
%  \fax{+44 (0)1603 592044}
%  \web{www.uea.ac.uk}
%  \position{Senior Research Associate}
%  \begin{document}
%  \begin{letter}
%  {Mr.~A.N.~Other \\ Some Company \\ Some Street \\ Sometown}
%  \subject{A demonstration letter}
%  \opening{Dear Mr.~Other,}
%
%  This is a rather boring letter, which simply shows how to use
%  the class file.
%
%  \closing{Yours faithfully,}
%  \end{letter}
%  \end{document}
%\end{LaTeXexample}
%
% To make configuration easier, the class will attempt to load a
% configuration file \file{uealttr.cfg}.  This can be used to set up
% repeated data. This can also contain other instructions for \LaTeX.
%  For example, to include the standard data above in every letter,
% the class author uses a configuration file reading
%\begin{LaTeXexample}[noexample]
%  \name{Joseph Wright}
%  \faculty{Faculty of Science}
%  \department{School of Chemical Sciences and Pharmacy}
%  \address{
%    University of East Anglia \\
%    Norwich NR4 7TJ \\
%    United Kingdom}
%  \email{joseph.wright@uea.ac.uk}
%  \phone{+44 (0)1603 591680}
%  \fax{+44 (0)1603 592044}
%  \web{www.uea.ac.uk}
%  \position{Senior Research Associate}
%  \RequirePackage{microtype}
%  \RequirePackage{babel}
%\end{LaTeXexample}
%
% The letter can then be reduced to.
%\begin{LaTeXexample}[noexample]
%  \documentclass[english,UKenglish]{uealttr}
%  \begin{document}
%  \begin{letter}
%  {Mr.~A.N.~Other \\ Some Company \\ Some Street \\ Sometown}
%  \subject{A demonstration letter}
%  \opening{Dear Mr.~Other,}
%
%  This is a rather boring letter, which simply shows how to use
%  the class file.
%
%  \closing{Yours faithfully,}
%  \end{letter}
%  \end{document}
%\end{LaTeXexample}
%
%\StopEventually{%
%  \PrintChanges
%  \PrintIndex}
%
%\iffalse
%<*class>
%\fi
%
%\section{The code}
%\begin{macro}{\uea@id}
% The package starts with the usual identification.
%    \begin{macrocode}
\NeedsTeXFormat{LaTeX2e}
\def\uea@id$#1: #2.#3 #4 #5-#6-#7 #8 #9${#5/#6/#7}
\ProvidesClass{uealttr}
  [\uea@id$Id: uealttr.dtx 14 2008-07-23 22:03:33Z joseph $
   v1.0a A letter class for UEA]
%    \end{macrocode}
%\end{macro}
% The standard support packages are loaded.
%    \begin{macrocode}
\LoadClass[10pt,a4paper]{letter}
\RequirePackage[T1]{fontenc}
\RequirePackage[final]{graphicx}
\RequirePackage[parfill]{parskip}
\RequirePackage{helvet,eso-pic,ifpdf,kvoptions}
\RequirePackage[
  hmargin=30mm,
  vmargin=25mm,
  dvips]{geometry}
%    \end{macrocode}
%\begin{macro}{\ifuea@personal}
%\begin{macro}{\ifuea@confidential}
%\begin{macro}{\ifuea@logo}
%\begin{macro}{\ifuea@draft}
% The single package option is declared.
%    \begin{macrocode}
\SetupKeyvalOptions{
  family = uea,
  prefix = uea@}
\DeclareBoolOption{personal}
\DeclareBoolOption{confidential}
\DeclareBoolOption{logo}
\DeclareBoolOption{draft}
\DeclareComplementaryOption{final}{draft}
\setkeys{uea}{
  personal = false,
  confidential = false,
  logo = true,
  draft = false}
\ProcessKeyvalOptions{uea}
\ifuea@draft
  \setlength\overfullrule{5pt}
\else
  \setlength\overfullrule{0pt}
\fi
%    \end{macrocode}
%\end{macro}
%\end{macro}
%\end{macro}
%\end{macro}
%\begin{macro}{\uea@pandc}
% For personal and confidential letters, some text is set up here.
%    \begin{macrocode}
\newcommand*{\uea@pandc}{}
\ifuea@personal
  \renewcommand*{\uea@pandc}{PERSONAL}
  \ifuea@confidential
    \renewcommand*{\uea@pandc}{PERSONAL \& CONFIDENTIAL}
  \fi
\else
  \ifuea@confidential
    \renewcommand*{\uea@pandc}{CONFIDENTIAL}
  \fi
\fi
%    \end{macrocode}
%\end{macro}
% The UEA style is to use Helvetica for all text.
%    \begin{macrocode}
\renewcommand{\familydefault}{\sfdefault}
%    \end{macrocode}
%\begin{macro}{\subject}
%\begin{macro}{\lettersubject}
% The subject of the letter is set up.
%    \begin{macrocode}
\newcommand*{\subject}[1]{\def\lettersubject{#1}}
\subject{}
%    \end{macrocode}
%\end{macro}
%\end{macro}
%\begin{macro}{\faculty}
%\begin{macro}{\fromfaculty}
%\begin{macro}{\department}
%\begin{macro}{\fromdept}
% A number of macros are used to store the various pieces of data
% used in the address block.  First the faculty and department.
%    \begin{macrocode}
\newcommand*{\faculty}[1]{\def\fromfaculty{#1}}
\newcommand*{\department}[1]{\def\fromdept{#1}}
\faculty{}
\department{}
%    \end{macrocode}
%\end{macro}
%\end{macro}
%\end{macro}
%\end{macro}
%\begin{macro}{\email}
%\begin{macro}{\fromemail}
%\begin{macro}{\phone}
%\begin{macro}{\fromphone}
%\begin{macro}{\fax}
%\begin{macro}{\fromfax}
%\begin{macro}{\web}
%\begin{macro}{\fromweb}
% Next come the contact details for the right-hand area.
%    \begin{macrocode}
\newcommand*{\email}[1]{\def\fromemail{#1}}
\newcommand*{\phone}[1]{\def\fromphone{#1}}
\newcommand*{\fax}[1]{\def\fromfax{#1}}
\newcommand*{\web}[1]{\def\fromweb{#1}}
\email{}
\phone{}
\fax{}
\web{}
%    \end{macrocode}
%\end{macro}
%\end{macro}
%\end{macro}
%\end{macro}
%\end{macro}
%\end{macro}
%\end{macro}
%\end{macro}
%\begin{macro}{\position}
%\begin{macro}{\fromposition}
%\begin{macro}{\signphone}
%\begin{macro}{\fromsignphone}
%\begin{macro}{\signemail}
%\begin{macro}{\fromsignemail}
% Finally, some details added after the signature.
%    \begin{macrocode}
\newcommand*{\position}[1]{\def\fromposition{#1}}
\newcommand*{\signphone}[1]{\def\fromsignphone{#1}}
\newcommand*{\signemail}[1]{\def\fromsignemail{#1}}
\position{}
\signphone{}
\signemail{}
%    \end{macrocode}
%\end{macro}
%\end{macro}
%\end{macro}
%\end{macro}
%\end{macro}
%\end{macro}
%\begin{macro}{\logo}
%\begin{macro}{\fromlogo}
% Any local configuration is loaded, and options are processed.
%    \begin{macrocode}
\newcommand*{\logo}[1]{\def\fromlogo{#1}}
\logo{uealogo}
\InputIfFileExists{uealttr.cfg}
  {\ClassInfo{uealttr}{Loaded local configuration file}}
  {}
%    \end{macrocode}
%\end{macro}
%\end{macro}
% If no logo is available, the package will skip trying to position
% it.
%    \begin{macrocode}
\ifpdf
  \IfFileExists{\fromlogo.pdf}
    {}
    {\ifuea@logo
       \ClassWarning{uealttr}
         {Logo file \fromlogo.pdf not found!\MessageBreak
          No logo will be included in output}
     \fi
     \uea@logofalse}
\else
  \IfFileExists{\fromlogo.eps}
    {}
    {\ifuea@logo
       \ClassWarning{uealttr}
         {Logo file \fromlogo.eps not found!\MessageBreak
          No logo will be included in output}
     \fi
     \uea@logofalse}
\fi
%    \end{macrocode}
%\begin{environment}{wider}
% A trick taken from \pkg{memoir} and the UK FAQ: here, the extra
% width needed is known.
%    \begin{macrocode}
\newenvironment{wider}{%
  \begin{list}{}{%
    \setlength{\topsep}{0pt}%
    \setlength{\leftmargin}{\z@}%
    \setlength{\rightmargin}{-22mm}%
    \setlength{\listparindent}{\parindent}%
    \setlength{\itemindent}{\parindent}%
    \setlength{\parsep}{\parskip}%
  }%
  \item[]}{\end{list}}

%    \end{macrocode}
%\end{environment}
%\begin{macro}{\opening}
% To achieve the correct layout, completely new \cs{opening}
% and \cs{closing} macros are employed.
%    \begin{macrocode}
\renewcommand*{\opening}[1]{%
  \thispagestyle{empty}%
  \vspace*{19mm}%
%    \end{macrocode}
% If the logo has been requested, it is included here.
%    \begin{macrocode}
  \ifuea@logo
    \AddToShipoutPicture*{%
      \setlength{\unitlength}{1mm}%
      \put(147,257){\includegraphics{\fromlogo}}}%
  \fi
%    \end{macrocode}
% The address block is created by using a series of minipages of the
% correct width.  All of this is nested inside a group where the
% margins are made wider.
%    \begin{macrocode}
  \begin{wider}
    \setlength{\leftmargin}{\z@}%
    \setlength{\rightmargin}{22mm}%
    \begin{minipage}[t]{80mm}
      \begin{minipage}[t]{80mm}
        \raggedright
        \ifx\@empty\uea@pandc\else
          \textbf{\uea@pandc} \\*[\baselineskip]
        \fi
        \toname \\
        \toaddress
      \end{minipage}%
%    \end{macrocode}
% A zero-width minipage is used to force the date downward to the
% desired position.
%    \begin{macrocode}
      \begin{minipage}[t]{0mm}
        \vspace*{35mm}%
      \end{minipage}
      \@date
    \end{minipage}%
%    \end{macrocode}
% This minipage generates the whitespace between the two address
% blocks.
%    \begin{macrocode}
    \begin{minipage}[t]{42mm}
      \hspace{42mm}%
      \vspace*{55mm}%
    \end{minipage}%
%    \end{macrocode}
% For the address block, a series of checks are made so that no empty
% lines are ended with |\\|.
%    \begin{macrocode}
    \begin{minipage}[t]{50mm}
      \footnotesize
      \raggedright
      \ifx\@empty\fromfaculty
        \ifx\@empty\fromdept\else
          \ignorespaces\fromdept \\*[\baselineskip]
        \fi
      \else
        \ifx\@empty\fromdept
          \textbf{\ignorespaces\fromfaculty} \\*[\baselineskip]
        \else
          \textbf{\ignorespaces\fromfaculty} \\
          \ignorespaces\fromdept \\*[\baselineskip]
        \fi
      \fi
      \ifx\@empty\fromaddress\else
        \ignorespaces\fromaddress \\*[\baselineskip]
      \fi
      \ifx\@empty\fromemail\else
        Email: \ignorespaces\fromemail \\
      \fi
      \ifx\@empty\fromphone\else
        Tel: \ignorespaces\fromphone \\
      \fi
      \ifx\@empty\fromfax\else
        Fax: \ignorespaces\fromfax \\
      \fi
      \ifx\@empty\fromweb\else
        \ignorespaces\fromweb \\
      \fi
    \end{minipage}%
  \end{wider}
  \par\noindent#1\par\nobreak
  {\bfseries
   \ifx\@empty\lettersubject\else
     \expandafter\uea@MakeUppercase\lettersubject \\\@empty
   \fi}}
%    \end{macrocode}
%\end{macro}
%\begin{macro}{\uea@MakeUppercase}
% A support macro is needed to allow multi-line uppercase.
%    \begin{macrocode}
\def\uea@MakeUppercase#1\\#2\@empty{%
  \MakeUppercase{#1}%
  \ifx\@empty#2\@empty\else
    \\ \expandafter\uea@MakeUppercase#2\@empty
  \fi}
%    \end{macrocode}
%\end{macro}
%\begin{macro}{\closing}
%\darg{closing}
% The same tricks are used for the closing.
%    \begin{macrocode}
\renewcommand{\closing}[1]{%
  \par\nobreak\vspace{\parskip}%
  \stopbreaks
  \noindent
  \ignorespaces #1\\[24mm]
  \ifx\@empty\fromsig
    \fromname \\
  \else
    \fromsig \\
  \fi
  \ifx\@empty\fromposition\else
    \fromposition \\
  \fi
  \ifx\@empty\fromsignphone\else
    \fromsignphone \\
  \fi
  \ifx\@empty\fromsignemail\else
    \fromsignemail \\
  \fi
  \strut
  \par}
%    \end{macrocode}
%\end{macro}
%\begin{macro}{\stopletter}
% The \cs{stopletter} macro is used to push the contents up the page,
% in contrast to normal \LaTeX\ behaviour.
%    \begin{macrocode}
\renewcommand*{\stopletter}{\vfill}
%    \end{macrocode}
%\end{macro}
%
%\Finale
%\iffalse
%</class>
%<*jawltxdoc>
\NeedsTeXFormat{LaTeX2e}
\ProvidesPackage{jawltxdoc}
\usepackage[T1]{fontenc}
\usepackage{lmodern}
\usepackage[final]{listings,graphicx,microtype}
\usepackage[scaled=0.95]{helvet}
\usepackage[version=3]{mhchem}
\usepackage[osf]{mathpazo}
\usepackage{booktabs,array,url,courier,xspace,varioref}
\usepackage{upgreek,ifpdf,float,caption,longtable,babel}
\begingroup
  \@ifundefined{eTeXversion}
    {\aftergroup\@gobble}
    {\aftergroup\@firstofone}
\endgroup
{\usepackage{etoolbox}}
\floatstyle{plaintop}
\restylefloat{table}
\labelformat{figure}{\figurename~#1}
\labelformat{table}{\tablename~#1}
\ifpdf
  \usepackage{embedfile}
  \embedfile[%
    stringmethod=escape,%
    mimetype=plain/text,%
    desc={LaTeX docstrip source archive for package `\jobname'}%
    ]{\jobname.dtx}
\fi
\IfFileExists{\jobname.sty}
  {\usepackage{\jobname}}{}
\usepackage[numbered]{hypdoc}
\setcounter{IndexColumns}{2}
\newlength\LaTeXwidth
\newlength\LaTeXoutdent
\newlength\LaTeXgap
\setlength\LaTeXgap{1em}
\setlength\LaTeXoutdent{-0.15\textwidth}
\newbox\lst@samplebox
\edef\LaTeXexamplefile{\jobname.tmp}
\lst@RequireAspects{writefile}
\lstnewenvironment{LaTeXexample}[1][example]{%
  \global\let\lst@intname\@empty
  \ifcsname LaTeXcode#1\endcsname
    \expandafter\let\expandafter\LaTeXcode
      \csname LaTeXcode#1\endcsname
    \expandafter\let\expandafter\LaTeXcodeend
      \csname LaTeXcode#1end\endcsname
  \else
    \PackageError{jawltxdoc}
      {Undefined example type `#1'}
      \@ehd
    \let\LaTeXcode\relax
    \let\LaTeXcodeend\relax
  \fi
  \LaTeXcode}
  {\lst@EndWriteFile
   \LaTeXcodeend}
\newcommand*{\LaTeXcodeexample}{%
  \setbox\lst@samplebox=\hbox\bgroup
  \LaTeXcodefloat}
\let\LaTeXcoderesultonly\LaTeXcodeexample
\newcommand*{\LaTeXcodeexampleend}{%
  \egroup
  \setlength\LaTeXwidth{\wd\lst@samplebox}%
  \begin{list}{}{%
    \setlength\itemindent{0pt}
    \setlength\leftmargin\LaTeXoutdent
    \setlength\rightmargin{0pt}}%
    \item
      \setlength\LaTeXoutdent{-0.15\textwidth}
      \begin{minipage}[c]{%
        \textwidth-\LaTeXwidth-\LaTeXoutdent-\LaTeXgap}
        \LaTeXcodefloatend
      \end{minipage}%
      \hfill
      \begin{minipage}[c]{\LaTeXwidth}%
        \hbox to\linewidth{\box\lst@samplebox\hss}%
      \end{minipage}%
  \end{list}}
\newcommand*{\LaTeXcodefloat}{%
  \setkeys{lst}{tabsize=4,gobble=3,breakindent=0pt,
    basicstyle=\small\ttfamily,basewidth=0.51em,
    keywordstyle=\color{blue}}%
  \lst@BeginAlsoWriteFile{\LaTeXexamplefile}}
\let\LaTeXcodenoexample\LaTeXcodefloat
\let\LaTeXcodenoexampleend\@empty
\newcommand*{\LaTeXcodefloatend}{%
  \MakePercentComment\catcode`\^^M=10\relax
  \small
  {\setkeys{lst}{SelectCharTable=\lst@ReplaceInput{\^\^I}%
    {\lst@ProcessTabulator}}%
    \leavevmode \input{\LaTeXexamplefile}}%
  \MakePercentIgnore}
\newcommand*{\LaTeXcoderesultonlyend}{\egroup\LaTeXcodefloatend}
\lstnewenvironment{BibTeXexample}{%
  \global\let\lst@intname\@empty
  \setbox\lst@samplebox=\hbox\bgroup
  \setkeys{lst}{tabsize=4,gobble=3,breakindent=0pt,
    basicstyle=\small\ttfamily,basewidth=0.51em,
    keywordstyle=\color{black}}
  \lst@BeginAlsoWriteFile{\LaTeXexamplefile}}
 {\lst@EndWriteFile
   \LaTeXcodeexampleend}
\newcommand*{\DescribeOption}{%
  \leavevmode\@bsphack\begingroup\MakePrivateLetters
  \Describe@Option}
\newcommand*{\Describe@Option}[1]{\endgroup
              \marginpar{\raggedleft\PrintDescribeEnv{#1}}%
              \SpecialOptionIndex{#1}\@esphack\ignorespaces}
\newcommand*{\SpecialOptionIndex}[1]{\@bsphack
    \index{#1\actualchar{\protect\ttfamily#1}
           (option)\encapchar usage}%
    \index{options:\levelchar#1\actualchar{\protect\ttfamily#1}%
      \encapchar usage}\@esphack}
\newcommand*{\indexopt}[1]{\DescribeOption{#1}\opt{#1}}
\newcommand*{\DescribeOptionInfo}[2]{%
  \DescribeOption{#1}%
  \opt{#1=\meta{#2}}\xspace}
\newcommand*{\ofixarg}[1]{%
  {\ttfamily[}%
  \ifmmode \expandafter \nfss@text \fi
  {%
    \meta@font@select
    \edef\meta@hyphen@restore{%
      \hyphenchar\the\font\the\hyphenchar\font}%
    \hyphenchar\font\m@ne
    \language\l@nohyphenation
    #1\/%
    \meta@hyphen@restore
    }%
    {\ttfamily]}}
\newcommand*{\pkg}[1]{\textsf{#1}}
\newcommand*{\currpkg}{\pkg{\jobname}\xspace}
\newcommand*{\opt}[1]{\texttt{#1}}
\newcommand*{\defaultopt}[1]{\opt{\textbf{#1}}}
\newcommand*{\file}[1]{\texttt{#1}}
\newcommand*{\ext}[1]{\file{.#1}}
\newcommand*{\latin}[1]{\emph{#1}}
\newcommand*{\etc}{%
  \@ifnextchar.
    {\latin{etc}}
    {\latin{etc}.\xspace}}
\newcommand*{\eg}{%
  \@ifnextchar.
    {\latin{e.g}}
    {\latin{e.g}.\xspace}}
\newcommand*{\ie}{%
  \@ifnextchar.
    {\latin{i.e}}
    {\latin{i.e}.\xspace}}
\newcommand*{\etal}{%
  \@ifnextchar.
    {\latin{et~al.}}
    {\latin{et~al}.\xspace}}
\newcommand*{\AMS}{{\protect\usefont{OMS}{cmsy}{m}{n}%
  A\kern-.1667em\lower.5ex\hbox{M}\kern-.125emS}}
\providecommand*{\eTeX}{\ensuremath{\varepsilon}-\TeX}
\DeclareRobustCommand*{\XeTeX}
  {X\kern-.125em\lower.5ex\hbox{\reflectbox{E}}\kern-.1667em\TeX}
\providecommand*{\CTAN}{\textsc{ctan}}
\@ifpackageloaded{etoolbox}
  {\patchcmd{\@addmarginpar}
    {\@latex@warning@no@line {Marginpar on page \thepage\space moved}}
    {\relax}{}{}}
  {}
\newcounter{argument}
\g@addto@macro\endmacro{\setcounter{argument}{0}}
\newcommand*\darg[1]{%
  \stepcounter{argument}%
  {\ttfamily\char`\#\theargument~:~}#1\par\noindent\ignorespaces}
\newcommand*\doarg[1]{%
  \stepcounter{argument}%
  {\ttfamily\makebox[0pt][r]{[}%
   \char`\#\theargument]:~}#1\par\noindent\ignorespaces}
%</jawltxdoc>
%\fi
